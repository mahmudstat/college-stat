\documentclass[a4paper,oneside, margin=1.4in]{book}

\usepackage{lipsum}
\usepackage[hidelinks]{hyperref}
\usepackage{titletoc}
\usepackage{geometry}
\geometry{a4paper, margin=1in}
\titlecontents*{chapter}
  [0pt]% <left>
  {}
  {\chaptername\ \thecontentslabel\quad}
  {}
  {\bfseries\hfill\contentspage}    

\usepackage{bookmark}
\usepackage{etoolbox}

\makeatletter
\newcommand*{\AddChapterPrefixInBookmarks}{%
  \if@mainmatter
    \ifnum\bookmarkget{level}=0 %
      \preto\bookmark@text{\@chapapp\space}%
    \fi
  \fi
}
\makeatother

\bookmarksetup{
  numbered,
  addtohook=\AddChapterPrefixInBookmarks,
}

% Workaround for numbered sections in unnumbered
% chapter "Introduction" to avoid chapter number
% zero.
\renewcommand*{\thesection}{%
  \ifcase\value{chapter}%
  \else
    \thechapter.%
  \fi
  \arabic{section}%
}

\title{My document}

\begin{document}
\frontmatter

\begin{titlepage}
    \begin{center}
        \vspace*{1cm}
            
        \Huge
        \textbf{Statistics Practical Works}
            
        \vspace{0.5cm}
        \LARGE
        Paper
            
        \vspace{1.5cm}
            
        \textbf{Abdullah Al Mahmud}
            
        \vfill
            
            
        \vspace{0.8cm}
            
            
        \Large
        www.statmania.info\\
            
    \end{center}
\end{titlepage}


\tableofcontents


\mainmatter

\part{First Paper}
\chapter{Chapter} 
\section{Construction of Frequency Distribution}

\subsection{Problem}

The following information represent amount of rainfall in 30 districts in Bangladesh. \\

360 442 452 438 449 387 383 348 315 347 308 313 494 477 413 \\
335 495 420 424 396 322 398 327 349 316 488 496 443 337 460 \\

Make a frequency distribution from the data and interpret. 

\subsection{Solution}
\subsection{Theory}

To make a frequency distribution, we require three things. 

i. Range, R = Highest Value - Lowest value \\
ii. Number of class, $k=1+3.322log_{10}N$ \\
iii. Class Interval, $CI = \frac{R}{k}$

\section{Determination of Arithmetic Mean, Combined Arithmetic Mean, Geometric Mean, Harmonic Mean from Grouped and Ungrouped Data.}

\subsection{Problem}

1. Heights (in cm) of 10 students in a class are given below:
\begin{center}
169, 172, 159, 165, 164, 170, 172, 175, 180, 150
\end{center}

Find AM, GM, and HM. 


2. Income of 100 individuals are collected (in thousand BDT)


\begin{table}[h]
\centering
\begin{tabular}{|c|c|c|c|c|c|}
\hline
Income Level & 10-20 & 20-30 & 30-50 & 50-80 & 80-120 \\ \hline
Frequency & 35 & 24 & 20 & 16 & 5 \\ \hline
\end{tabular}
\end{table}

Find AM, GM, and HM. 

\subsection{Solution}

\subsubsection{Theory}

Formula. 

\begin{table}[h]
\centering
\begin{tabular}{c|c|c}
Type & Grouped & Ungrouped \\ \hline
AM &  &  \\ \hline
GM &  &  \\ \hline
HM &  & 
\end{tabular}
\end{table}

\subsubsection{Equipment/Tools}

\subsubsection{Computation}

\subsubsection{Construction}

\subsubsection{Interpretation}

\subsubsection{Precaution}

\backmatter
\chapter{Conclusion}
\lipsum[8]

\tableofcontents
\end{document}