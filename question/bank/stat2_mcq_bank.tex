\documentclass{exam}
%\documentclass[11pt,a4paper]{exam}
\usepackage{amsmath,amsthm,amsfonts,amssymb,dsfont}
\usepackage{ifthen}
\usepackage{enumerate}% http://ctan.org/pkg/enumerate
\usepackage{multicol}

% Accumulate the answers. Unmodified from Phil Hirschorn's answer
% https://tex.stackexchange.com/questions/15350/showing-solutions-of-the-questions-separately/15353
\newbox\allanswers
\setbox\allanswers=\vbox{}

\newenvironment{answer}
{%
    \global\setbox\allanswers=\vbox\bgroup
    \unvbox\allanswers
}%
{%
    \bigbreak
    \egroup
}

\newcommand{\showallanswers}{\par\unvbox\allanswers}
% End Phil's answer


% Is there a better way?
\newcommand*{\getanswer}[5]{%
    \ifthenelse{\equal{#5}{a}}
    {\begin{answer}\thequestion. (a)~#1\end{answer}}
    {\ifthenelse{\equal{#5}{b}}
        {\begin{answer}\thequestion. (b)~#2\end{answer}}
        {\ifthenelse{\equal{#5}{c}}
            {\begin{answer}\thequestion. (c)~#3\end{answer}}
            {\ifthenelse{\equal{#5}{d}}
                {\begin{answer}\thequestion. (d)~#4\end{answer}}
                {\begin{answer}\textbf{\thequestion. (#5)~Invalid answer choice.}\end{answer}}}}}
}

\setlength\parindent{0pt}
%usage \choice{ }{ }{ }{ }
%(A)(B)(C)(D)
\newcommand{\fourch}[5]{
    \par
    \begin{tabular}{*{4}{@{}p{0.23\textwidth}}}
        (a)~#1 & (b)~#2 & (c)~#3 & (d)~#4
    \end{tabular}
    \getanswer{#1}{#2}{#3}{#4}{#5}
}

%(A)(B)
%(C)(D)
\newcommand{\twoch}[5]{
    \par
    \begin{tabular}{*{2}{@{}p{0.46\textwidth}}}
        (a)~#1 & (b)~#2
    \end{tabular}
    \par
    \begin{tabular}{*{2}{@{}p{0.46\textwidth}}}
        (c)~#3 & (d)~#4
    \end{tabular}
    \getanswer{#1}{#2}{#3}{#4}{#5}
}

%(A)
%(B)
%(C)
%(D)
\newcommand{\onech}[5]{
    \par
    (a)~#1 \par (b)~#2 \par (c)~#3 \par (d)~#4
    \getanswer{#1}{#2}{#3}{#4}{#5}
}

\newlength\widthcha
\newlength\widthchb
\newlength\widthchc
\newlength\widthchd
\newlength\widthch
\newlength\tabmaxwidth

\setlength\tabmaxwidth{0.96\textwidth}
\newlength\fourthtabwidth
\setlength\fourthtabwidth{0.25\textwidth}
\newlength\halftabwidth
\setlength\halftabwidth{0.5\textwidth}

\newcommand{\choice}[5]{%
\settowidth\widthcha{AM.#1}\setlength{\widthch}{\widthcha}%
\settowidth\widthchb{BM.#2}%
\ifdim\widthch<\widthchb\relax\setlength{\widthch}{\widthchb}\fi%
    \settowidth\widthchb{CM.#3}%
\ifdim\widthch<\widthchb\relax\setlength{\widthch}{\widthchb}\fi%
    \settowidth\widthchb{DM.#4}%
\ifdim\widthch<\widthchb\relax\setlength{\widthch}{\widthchb}\fi%

% These if statements were bypassing the \onech option.
% \ifdim\widthch<\fourthtabwidth
%     \fourch{#1}{#2}{#3}{#4}{#5}
% \else\ifdim\widthch<\halftabwidth
% \ifdim\widthch>\fourthtabwidth
%     \twoch{#1}{#2}{#3}{#4}{#5}
% \else
%      \onech{#1}{#2}{#3}{#4}{#5}
%  \fi\fi\fi}

% Allows for the \onech option.
\ifdim\widthch>\halftabwidth
    \onech{#1}{#2}{#3}{#4}{#5}
\else\ifdim\widthch<\halftabwidth
\ifdim\widthch>\fourthtabwidth
    \twoch{#1}{#2}{#3}{#4}{#5}
\else
    \fourch{#1}{#2}{#3}{#4}{#5}
\fi\fi\fi}



\begin{document}
\begin{titlepage}
    \begin{center}
        \vspace*{1cm}
            
        \Huge
        \textbf{Statistics MCQ Question Bank}
            
        \vspace{0.5cm}
        \LARGE
        Second Paper \\


            
        \vspace{1.5cm}
            

            
        \vfill
            
            
        \vspace{0.8cm}
            
                    \textbf{Abdullah Al Mahmud} \\
        \Large
        www.statmania.info\\
            
    \end{center}
\end{titlepage}

 \section{Introduction to Probability}


\begin{questions}

\question \textbf{An act repeated under some specific conditions is called --}
\choice{Event}{Experiment}{Sample}{Sample space}{b}

\question \textbf{$P(0)$ implies -- }
\choice{A certain event}{An uncertain event}{An impossible event}{A probable event}{c}

\question \textbf{Events having some common elements are called --}
\choice{Complementary events}{Mutually exclusive events}{Exhaustive events}{Non-Mutually exclusive events events}{a}

\question \textbf{The minimum value of probability is}
\choice{$-\alpha$}{1}{0}{-1}{c}

\subsection {Permutation-Combination}

\question \textbf{Three objects can be placed in 2 positions in -- ways.}
\choice{3}{4}{6}{8}{c}

\question \textbf{In how many ways can a team of 2 be formed from 4 people?}
\choice{4}{6}{8}{12}{b}

\question \textbf{$\displaystyle ^np_r=$}
\choice{$\displaystyle \frac {n!}{(n-r)!}$}{$\displaystyle \frac {n!}{(n+r)!}$}{$\displaystyle \frac {n!}{r!}$}{$\displaystyle \frac {n!}{(r-n)!}$}{a}

\question \textbf{$\displaystyle ^nC_r=$}
\choice{$\displaystyle \frac {n!}{(n-1)!(n+r)!}$}{$\displaystyle \frac {r!}{n!(n-r)!}$}{$\displaystyle \frac {n!(n-1)!}{r!}$}{$\displaystyle \frac {n!}{(r-n)!}$}{a}

\question \textbf{Each element of sample space is called--}
\choice{Trial}{Experiment}{Variable}{Sample Point}{d}

\question \textbf{Two events not ocurring together are called--}
\choice{dependent Events}{Independent Events}{Mutually Exclusive Events}{Marginal Events}{c}

\question \textbf{If A and B are independent, which formula is correct?}
\choice{$P(A \cap B) = P(A) \cdot P(B)$}{$P(A \cap B) = P(\bar A) \cdot P(B)$}{$P(A \cap B) = P(A) \cdot P(\bar B)$}{$P(A \cap \bar B) = P(A) \cdot P(B)$}{a}

\textbf{Answer the next three questions based on the following information.}

A card is drawn from of pack of playing cards.

\question \textbf{What is the probability that the card is a King?}
\choice{0.0192}{0.25}{0.5}{0.0769}{d}

\question \textbf{P(The card is not from Diamonds)--}
\choice{$\frac12$}{$0$}{$\frac34$}{$\frac14$}{c}

\question \textbf{P(The card is red or Clubs)}
\choice{$\frac14$}{$\frac12$}{$\frac23$}{$\frac34$}{d}

\question \textbf{If a neutral die is thrown, the probability of having a digit greater than 6 is}
\choice{$\frac 1 6$}{$\frac 0 6$}{$\frac 2 3$}{$\frac 3 6$}{b}

\question \textbf{Tossing a coin twice generates how many outcomes?}
\choice{4}{16}{8}{2}{a}

\question \textbf{The probability of two disjoint sets happening together is:}
\choice{0.5}{0}{1}{$0  \leq x < 1$}{b}

\textbf{Answer the next three questions using the following information}

$P(A) = \frac 1 3, P(B) = \frac 1 2 \space \& \space P(A\cup B) = \frac 7 {12}$

\question \textbf{$P(A \cap B) = ?$}
\choice{$\frac {5}{12}$}{$\frac12$}{$\frac{1}{4}$}{$\frac{15}{16}$}{c}

\question \textbf{$P(A \cap \bar B)=?$}
\choice{$\frac{1}{4}$}{$\frac{3}{4}$}{$\frac{5}{6}$}{$\frac{1}{12}$}{a}

\question \textbf{What is the probability that B occurs or A does not occur?}
\choice{$\frac{3}{4}$}{$\frac{7}{12}$}{$\frac{5}{12}$}{$\frac{11}{12}$}{d}

\question \textbf{An un contains 10 red and 5 black balls. Two balls are drawn; what is the probability of getting two red balls?}
\choice{$\frac 37$}{$\frac 47$}{$\frac {20}{21}$}{$\frac 2{21}$}{a}

\section{Random Variables}

\question \textbf{How many conditions does a probability density function have?}
\choice{2}{3}{4}{5}{b}

\question \textbf{The conditions of a probability distribution are--}

i. $\sum P(X) = 1$

ii. $\sum P(X) = 0$

iii. $0 \le P(X) \le 1$

\choice{i and ii}{i and iii}{ii and iii}{i, ii and iii}{b}

\textbf{Answer the next two questions using the following information}

\begin{table}[h]
	    \centering
\begin{tabular}{ccccccl}
x    & 1 & 2  & 3  & 4  & 5  & 6  \\ \hline
P(x) & k & 2k & 3k & 4k & 5k & 6k
\end{tabular}
\end{table}

\question \textbf{What is the value of k?}
\choice{$\frac{7}{21}$}{$\frac{5}{21}$}{$\frac{1}{21}$}{$1$}{c}

\question \textbf{What is the type of variable X?}
\choice{Discrete}{Discrete random}{Continuous}{Continuous random}{b}

\question \textbf{What is $F(\infty)$ for a distribution function $F(x)$?}
\choice{$-\infty$}{-1}{0}{1}{d}

\question \textbf{What is $F(-\infty)$ for a distribution function $F(x)$?}
\choice{$-\infty$}{-1}{0}{1}{c}

\question \textbf{How many types of random variables are there?}
\choice{2}{3}{4}{5}{a}

\textbf{Answer the next two questions using the following information}

$\displaystyle P(x) = \frac{x+1}{k}; x = 1,2,3,4$

\question \textbf{What is the value of k?}
\choice{10}{11}{14}{15}{c}

\question \textbf{$P(x)$ is a --}
\choice{Joint probability distribution}{Cumulative probability distribution}{Probability mass function}{Probability Density function}{c}

\question \textbf{The example of a discrete random variable is--}

i. Binomial variate 

ii. Poisson variate

iii. Normal variate 

\textbf{Which one is correct?}

\choice{i and ii}{i and iii}{ii and iii}{i, ii and iii}{a} 

\question \textbf{Which of the following is not a discrete random variable?}
\choice{umber of students}{Weight}{Number of heads in coin toss}{Population}{b}

\question \textbf{Which one is a property of a probability distribution?}
\choice{$P(x_i) = 0$}{$P(x_i \ne 1)$}{$\Sigma P(x_i) = 1$}{$\int_x P(X) dx \le 1$}{c}

\question \textbf{$f(x) = 2x; 0 <X<3$; What is F(3)?}
\choice{3}{0}{1}{0}{c}

\textbf{Answer the next two questions based on the following information:}

$P(x,y) = \frac 1{21}(x+y); x = 1,2,3$ and $y=1,2$

\question \textbf{P(x)=?}
\choice{$P(x) = \frac{2x+3}{21}$}{$P(x) = \frac{x+3}{27}$}{$P(x) = \frac{4x+3}{21}$}{$P(x) = \frac{2x+5}{21}$}{a}

\question \textbf{P(y)=?}
\choice{$\frac{y+2}{7}$}{$\frac{y+3}{7}$}{$\frac{3y+2}{7}$}{$\frac{y+2}{9}$}{c}

\question \textbf{Which one is not a discrete random variable?}
\choice{Number of studnets}{Weight}{Number of heads in five coin tosses}{Released version number of a software}{d}

\question \textbf{Which one is a property of joint probability distribution?}
\choice{$P(X_i,Y_j)<1$}{$P(X_i,Y_j)=0$}{$P(X_i,Y_j)<0$}{$0 \leq P(X_i,Y_j)\leq 1$}{d}

\question \textbf{If $f(x) = kx^3; -1 \leq x \leq 1$, then k is}

i) positive \\
ii) negative  \\
iii) lies from -1 to 1

\choice{i}{ii}{iii}{i and ii}{a}

\textbf{Answer the next two questions based on the following information.}

\begin{table}[h]
\begin{center}
\begin{tabular}{|l|l|l|l|l|l|l|}
\hline
x    & 4         & 5         & 6         & 3         & 2         & 1         \\ \hline
P(X) & $\frac16$ & $\frac16$ & $\frac16$ & $\frac16$ & $\frac16$ & $\frac16$ \\ \hline
\end{tabular}
\end{center}
\end{table}

\question \textbf{The value of $P(3<X<5)$ is:}
\choice{$\frac12$}{$\frac16$}{$\frac13$}{0}{b}

\question \textbf{$P(x \neq 2) is:$}
\choice{$\frac56$}{$0$}{1}{Can't be found from this information}{a}

\section{Mathematical Expectation}

\question \textbf{What is the expected value of of the squared deviation of the value of the random variable from their mean?}
\choice{Arithmetic Mean}{Expectation}{Variance}{Co-variance}{c}

\question \textbf{What is the minimum value of variance a random variable?}
\choice{$-\infty$}{1}{0}{-1}{c}

\question \textbf{If $y=ax+b$, what is the value of $V(y)?$}
\choice{$aV(X)$}{$a^2V(X)$}{$V(X)$}{$a^2$}{b}

\question \textbf{If $y=ax+b$, what is the value of $E(y)?$}
\choice{$aE(X) + b$}{$a^2E(X)$}{$E(X)$}{$b$}{a}

\question \textbf{What is the value of $V(5)$?}
\choice{0}{25}{5}{1}{a}

\question \textbf{If $P(x) = \frac 1n; x = 1,2,3,\cdots ,n$, what is the value of $E(X)?$}
\choice{$\frac n2$}{$\frac{n-1}{2}$}{$\frac{n+1}{2}$}{$n+1$}{c}

\question \textbf{If $P(x)= \frac{4-|5-x|}{k}; x=2,3,4, \cdots 8$, what is the value of k?}
\choice{5}{8}{16}{24}{c}

\question \textbf{Expected value of a constant a is --}
\choice{1}{Variance}{a}{a+1}{c}

\question \textbf{The variance of a constant m is --}
\choice{0}{1}{m}{$m^2$}{a}

\question \textbf{What is $V(X-Y) equal to?$}
\choice{$V(X)+V(Y)$}{$V(X)+V(Y)-2 Cov(X,Y)$}{$V(X)-V(Y)$}{$V(X)+V(Y)+2Cov(X,Y)$}{c}

\question \textbf{What is the value of V(2X+5)?}
\choice{$4V(X)-5$}{20}{$4V(X)$}{0}{c}

\question \textbf{If $P(x) = \frac 1{20}; x=1,2,3, \cdots,20,$ what is the standard deviation?}
\choice{1}{5.77}{7.75}{12.57}{a}

\question \textbf{Expectation measures --}
\choice{Dispersion}{Skewness}{Kurtosis}{Central tendency}{d}

\question \textbf{If $E(X) = -0.5$, then $E(1-2X) = $?}
\choice{0}{-1}{2}{1}{c}

\question \textbf{If $P(X) = \frac{1}{10}; x = 1,2,\cdots 10$, then $E(X) =$?}
\choice{10}{5.5}{0}{11}{b}

\question \textbf{Which formula of variance is correct?}
\choice{$V(X+Y) = V(X)+V(Y)-2Cov(X,Y)$}{$V(X+Y) = V(X)+V(Y)+2Cov(X,Y)$}{$V(X+Y) = V(X)+V(Y)-2Cov(X,Y)$}{$V(X+Y) = V(X)-V(Y)+2Cov(X,Y)$}{b}

\question \textbf{X is a constant; what is the value of $V(\frac X2)$?}

i) 0 \\
ii) $\frac12$ \\
iii) $\frac14$

\choice{ii}{i}{iii}{i and iii}{b}

\question \textbf{If $E(X)=2, E(X^2) = 8, V(X)= --$}
\choice{0}{2}{4}{8}{c}

\question \textbf{If $P(x)= \frac{4-|5-x|}{k}; x=2,3,4, \cdots 8$, what is the value of $E(X)$?}
\choice{3}{8}{16}{5}{d}

\question \textbf{If $P(x)= \frac{6-|7-x|}{k}; x=2,3,4, \cdots 12$, what is the value of $E(X)$?}
\choice{6}{9}{13}{36}{d}

\question \textbf{If $P(x)= \frac{3-|4-x|}{k}; x=2,3,4, \cdots 6$, what is the value of k?}
\choice{6}{9}{10}{40}{b}

\question \textbf{If the variance of X is 3, what is the variance of V(3)?}
\choice{1}{2}{3}{0}{d}

\question \textbf{If $V(X) = 5,$, what is $V(X+5)?$}
\choice{0}{5}{10}{25}{b}

\question \textbf{If $V(X) = 5,$, what is $V(2X+5)?$}
\choice{20}{5}{10}{25}{a}

\section{Binomial Distribution}

\question \textbf{How many parameters are there in a binomial distribution?}
\choice{1}{2}{3}{4}{b}

\question \textbf{What is the Mean of Binomial Distribution?}
\choice{np}{npq}{nq}{$\sqrt{npq}$}{a}

\question \textbf{What is the Variance of Binomial Distribution?}
\choice{np}{npq}{nq}{$\sqrt{npq}$}{b}

\question \textbf{What is the Standard Deviation of Binomial Distribution?}
\choice{np}{npq}{nq}{$\sqrt{npq}$}{d}

\question \textbf{What is the Coefficient of Variation of Binomial Distribution?}
\choice{np}{npq}{$\frac{q}{np}$}{$\sqrt{npq}$}{c}

\question \textbf{Which is true of mean (np) of Binomial Distribution?}
\choice{$np=0$}{$np<0$}{$np>0$}{$np\ne0$}{c}

\question \textbf{In a Binomial distribution, how are mean and variance related?}
\choice{$Mean > Variance$}{$Mean < Variance$}{$Mean = Variance$}{$Mean =2 \times Variance$}{a}

\question \textbf{When does Binomial distribution tend to Poisson distribution?}
\choice{$n \rightarrow \infty$ and $p \rightarrow \infty$}{$n \rightarrow 0$ and $p \rightarrow 0$}{$n \rightarrow \infty$ and $p \rightarrow 0$}{$n \rightarrow 0$ and $p \rightarrow \infty$}{c}

\textbf{Answer the next two questions based on the following information.}

X is a binomial variate with expectation 4 and standard deviation $\sqrt 3$.

\question \textbf{What are the values of the parameters (mean and probability)?}
\choice{$16, \frac 14$}{$16, \frac 34$}{$15, \frac 14$}{$10, \frac 14$}{a}

\question \textbf{What is $P(X \neq 0)?$}
\choice{0}{0.01}{0.99}{1}{c}

\section{Poisson Distribution}

\question \textbf{What is the mean of Poisson distribution}
\choice{$\frac 1{\sqrt m}$}{$m$}{$\frac 1m$}{$1+\frac 1m$}{b}

\question \textbf{Which relationship between mean and variance of Poisson Distribution is correct?}
\choice{$Mean > Variance$}{$Mean < Variance$}{$Mean = Variance$}{$Mean \ne Variance$}{c}

\question \textbf{What is the Variance of Poisson Distribution(with parameter m)?}
\choice{$\frac1{\sqrt{m}}$}{$\frac1m$}{$m$}{$\frac1{m+1}$}{c}

\question \textbf{What is the Standard Deviation of Poisson Distribution(with parameter m)?}
\choice{$\frac1{\sqrt{m}}$}{$\frac1m$}{$\sqrt{m}$}{$\frac1{m+1}$}{c}

\question \textbf{Which one is true of the parameter (m) of Poisson Distribution?}
\choice{$m=0$}{$m<0$}{$m>0$}{$m=1$}{c}

\question \textbf{The parameter of a Poisson Distribution is 5. What is its mean?}
\choice{2}{5}{2.24}{25}{b}

\question \textbf{When does Binomial Distribution tend to Poisson Distribution?}
\choice{$n \rightarrow \infty, p \rightarrow 0$ \& $np$ is finite}{$n \rightarrow \infty, p \rightarrow 0$ \& $np$ is infinite}{$n \rightarrow \infty, p 0 \infty$ \& $np$ is finite}{$n \rightarrow 0, p \rightarrow \infty$ \& $np$ is infinite}{a}

\question \textbf{The parameter of a Poisson variate is 2. What is its variance?}
\choice{0}{4}{$\sqrt 2$}{2}{d}

\question \textbf{X is a Poisson variate. P(2) = P(4). What is the value of the parameter?}
\choice{12}{3.46}{3.6}{4}{b}

\question \textbf{Mean of a Poisson variate is a. What is its standard deviation?}
\choice{0}{a}{$a^{\frac 12}$}{$a^2$}{c}

\section{Vital Statistics}

\question \textbf{Crude Birth Rate (CBR) is:}
\choice{$\frac BP \times 100$}{$\frac BP \times 1000$}{$\frac PB \times 100$}{$\frac FP \times 100$}{b}

\question \textbf{Which one is a measure of reproduction?}

i) CBR \\
ii) CDR \\
iii) NRR

\choice{i}{ii}{iii}{i and ii}{c}

\question \textbf{The number of people living per unit area is called--}
\choice{Population Index}{Population Density}{Human Development Index}{Dependency Ratio}{b}

\question \textbf{Which formula of GFR is accurate?}
\choice{$GFR = \frac{B}{P}\times 1000$}{$GFR = \frac{B}{F_{15-49}}\times 1000$}{$GFR = \frac{B_i}{F_i}\times 1000$}{$GFR = \frac{G_i}{F{15-49}}\times 1000$}{b}

\end{questions}

\newpage  %Uncomment to put on new age
%\bigskip

\begin{multicols}{3}
[
\textbf{Answer Key:}
]
\showallanswers % Phil Hirschorn
  \end{multicols}


\end{document}