\documentclass{article}
\usepackage{geometry}
\usepackage{amsfonts}

\geometry{
legalpaper, total={177.8mm, 290mm},left=20mm,
top=27mm, bottom=27mm,
}

\begin{document}

\begin{table}[h]
\centering
\begin{tabular}{lllll}
\textbf{\large SYLHET CADET COLLEGE} &  &  &  &  \\ \cline{4-5} 
FIRST TERM-END EXAMINATION - 2023 &  & \multicolumn{1}{l|}{} & \multicolumn{1}{l|}{Set} & \multicolumn{1}{l|}{:A} \\ \cline{4-5} 
CLASS: XI &  &  &  &  \\ \cline{3-5} 
STATISTICS (CREATIVE)& \multicolumn{1}{l|}{\textbf{Subject Code:}} & \multicolumn{1}{l|}{1} & \multicolumn{1}{l|}{2} & \multicolumn{1}{l|}{9} \\ \cline{3-5} 
 FIRST PAPER &  &  &  &  \\
TIME – 20 minutes &  &  &  &  \\
FULL MARKS – 20 &  &  &  & 
\end{tabular}
\end{table}
%  \normalfont\normalsize
 % 11.45a.m.~--~1.45p.m.

\hrule

\begin{center}
[\textbf{N.B.} – The figures of the right margin indicate full marks. Read the stems carefully and answer the associated questions. Answer any \textbf{FIVE} questions taking at least two questions from each group]\\


\end{center}
  \begin{enumerate}

  \item
	  \textbf{An analyst obtains some data:}
	  \begin{center}
	  $x_1=15, x_2=-12, x_3=17, x_4=11, x_5=23$
  \end{center}
  \begin{enumerate}
    \item
	What is sample? \hfill 1
    \item
	Briefly explain shift or origin and scale. \hfill 2
    \item  
	Compute the value of $\displaystyle \sum_{i=1}^5 (x_i-10)^2$ \hfill 3
    \item
	Find the value of $\displaystyle \sum_{i=1}^5 (5x_i^2-4x_i-3)$ and examine its dependency on origin and scale. \hfill 4
  \end{enumerate}
  
       \item
	  \textbf{Favorite colors of 30 individuals are noted down. There are five different colors. The recorded colors are given below:}
	  
	  \begin{centering}
	  Brown Red   Pink  Green Green Green Brown Pink  Brown Red   \\
    Brown Red   Green Pink  White Red   Brown Green White Brown \\
    White Brown Pink  Red   White Brown Green Red   Pink  Red  \\
    \end{centering}
    
  
  \begin{enumerate}
    \item
	What is nominal data? \hfill 1
    \item
	What are the ways to deal with categorical data? \hfill 2
    \item  
	Draw a Pie Chart from the above data and explain. \hfill 3
    \item
	Is Bar Diagram a better representation of this data? Justify. \hfill 4
  \end{enumerate}
  
   \item
	  \textbf{Grades of a an undergraduate student with major in statistics are given below:} 

	  [Credits serve as weights]

\begin{table}[h]
\centering
\begin{tabular}{c|c|c}
\hline
Course & Grade & Credit \\ \hline
Probability & 3.75 & 4 \\ 
Simulation & 3.50 & 3 \\ 
Calculas & 3.50 & 4 \\ 
Linear Algebra & 3.75 & 4 \\ 
Econometrics & 3.00 & 2 \\ 
Programming & 3.50 & 3 \\ \hline
\end{tabular}
\end{table}

  
  \begin{enumerate}
    \item
	Write down the formula of weighted mean. \hfill 1
    \item
	What is difference between weight and frequency? \hfill 2
    \item  
	Determine the GPA of the student. \hfill 3
    \item
	Determine the geometric mean for the data and evaluate \\ suitability. \hfill 4
  \end{enumerate}

\begin{center}
\textbf{\textit{Absence of evidence is not evidence of absence.} – Carl Sagan}
\end{center}
  
\end{enumerate}
\end{document}