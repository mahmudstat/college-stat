\documentclass{exam}
%\documentclass[11pt,a4paper]{exam}
\usepackage{amsmath,amsthm,amsfonts,amssymb,dsfont}
\usepackage{ifthen}
\usepackage[legalpaper, total={177.8mm, 290mm}]{geometry}
\usepackage{enumerate}% http://ctan.org/pkg/enumerate
\usepackage{multicol}
\usepackage{graphicx}



% Accumulate the answers. Unmodified from Phil Hirschorn's answer
% https://tex.stackexchange.com/questions/15350/showing-solutions-of-the-questions-separately/15353
\newbox\allanswers
\setbox\allanswers=\vbox{}

\newenvironment{answer}
{%
    \global\setbox\allanswers=\vbox\bgroup
    \unvbox\allanswers
}%
{%
    \bigbreak
    \egroup
}

\newcommand{\showallanswers}{\par\unvbox\allanswers}
% End Phil's answer


% Is there a better way?
\newcommand*{\getanswer}[5]{%
    \ifthenelse{\equal{#5}{a}}
    {\begin{answer}\thequestion. (a)~#1\end{answer}}
    {\ifthenelse{\equal{#5}{b}}
        {\begin{answer}\thequestion. (b)~#2\end{answer}}
        {\ifthenelse{\equal{#5}{c}}
            {\begin{answer}\thequestion. (c)~#3\end{answer}}
            {\ifthenelse{\equal{#5}{d}}
                {\begin{answer}\thequestion. (d)~#4\end{answer}}
                {\begin{answer}\textbf{\thequestion. (#5)~Invalid answer choice.}\end{answer}}}}}
}

\setlength\parindent{0pt}
%usage \choice{ }{ }{ }{ }
%(A)(B)(C)(D)
\newcommand{\fourch}[5]{
    \par
    \begin{tabular}{*{4}{@{}p{0.23\textwidth}}}
        (a)~#1 & (b)~#2 & (c)~#3 & (d)~#4
    \end{tabular}
    \getanswer{#1}{#2}{#3}{#4}{#5}
}

%(A)(B)
%(C)(D)
\newcommand{\twoch}[5]{
    \par
    \begin{tabular}{*{2}{@{}p{0.46\textwidth}}}
        (a)~#1 & (b)~#2
    \end{tabular}
    \par
    \begin{tabular}{*{2}{@{}p{0.46\textwidth}}}
        (c)~#3 & (d)~#4
    \end{tabular}
    \getanswer{#1}{#2}{#3}{#4}{#5}
}

%(A)
%(B)
%(C)
%(D)
\newcommand{\onech}[5]{
    \par
    (a)~#1 \par (b)~#2 \par (c)~#3 \par (d)~#4
    \getanswer{#1}{#2}{#3}{#4}{#5}
}

\newlength\widthcha
\newlength\widthchb
\newlength\widthchc
\newlength\widthchd
\newlength\widthch
\newlength\tabmaxwidth

\setlength\tabmaxwidth{0.96\textwidth}
\newlength\fourthtabwidth
\setlength\fourthtabwidth{0.25\textwidth}
\newlength\halftabwidth
\setlength\halftabwidth{0.5\textwidth}

\newcommand{\choice}[5]{%
\settowidth\widthcha{AM.#1}\setlength{\widthch}{\widthcha}%
\settowidth\widthchb{BM.#2}%
\ifdim\widthch<\widthchb\relax\setlength{\widthch}{\widthchb}\fi%
    \settowidth\widthchb{CM.#3}%
\ifdim\widthch<\widthchb\relax\setlength{\widthch}{\widthchb}\fi%
    \settowidth\widthchb{DM.#4}%
\ifdim\widthch<\widthchb\relax\setlength{\widthch}{\widthchb}\fi%

% These if statements were bypassing the \onech option.
% \ifdim\widthch<\fourthtabwidth
%     \fourch{#1}{#2}{#3}{#4}{#5}
% \else\ifdim\widthch<\halftabwidth
% \ifdim\widthch>\fourthtabwidth
%     \twoch{#1}{#2}{#3}{#4}{#5}
% \else
%      \onech{#1}{#2}{#3}{#4}{#5}
%  \fi\fi\fi}

% Allows for the \onech option.
\ifdim\widthch>\halftabwidth
    \onech{#1}{#2}{#3}{#4}{#5}
\else\ifdim\widthch<\halftabwidth
\ifdim\widthch>\fourthtabwidth
    \twoch{#1}{#2}{#3}{#4}{#5}
\else
    \fourch{#1}{#2}{#3}{#4}{#5}
\fi\fi\fi}


\begin{document}

\begin{table}[h]
\centering
\begin{tabular}{lllll}
\textbf{\large SYLHET CADET COLLEGE} &  &  &  &  \\ \cline{4-5} 
PRE-TEST EXAMINATION - 2023 &  & \multicolumn{1}{l|}{} & \multicolumn{1}{l|}{Set} & \multicolumn{1}{l|}{C} \\ \cline{4-5} 
CLASS: XII &  &  &  &  \\ \cline{3-5} 
MULTIPLE CHOICE QUESTIONS & \multicolumn{1}{l|}{\textbf{Subject Code:}} & \multicolumn{1}{l|}{1} & \multicolumn{1}{l|}{3} & \multicolumn{1}{l|}{0} \\ \cline{3-5} 
STATISTICS SECOND PAPER &  &  &  &  \\
TIME – 25 minutes &  &  &  &  \\
FULL MARKS – 25 &  &  &  & 
\end{tabular}
\end{table}
%  \normalfont\normalsize
 % 11.45a.m.~--~1.45p.m.
\hrule

\begin{center}
[N.B. – Answer all the questions. Each question carries ONE mark. Block fully, with a black ball- point pen, the circle of the letter that stands for the correct/best answer in the “Answer sheet” for the Multiple Choice Questions Examination.]\\

  
  \textbf{Candidates are asked not to leave any mark or spot on the question paper.}
\end{center}
\begin{questions}

\question \textbf{Three objects can be placed in 2 positions in -- ways.}
\choice{3}{4}{6}{8}{c}

\question \textbf{A die is thrown twice. This is called --}
\choice{An experiment}{sample space}{A random experiment}{A trial}{c}

\question \textbf{A coin is thrown thrice. How many outcomes are generated?}
\choice{3}{4}{8}{9}{c}

\question \textbf{Which is the formula of empirical/relative frequency approach of probability?}
\choice{$P=\frac{\text{No. of favorable outcomes}}{\text{Total no. of possible outcomes}}$}{$P=\frac{\text{No. of total outcomes}}{\text{No. of favorable outcomes}}$}{$\displaystyle P = \lim_{n(S)\to\infty} \frac{n(A)}{n(S)}$}{$\displaystyle P = \lim_{n(A)\to\infty} \frac{n(A)}{n(S)}$}{a}

\question \textbf{What is the correct formula for conditional probability?}
\choice{$P(A|B) = \frac{P(A \cap B)}{P(B|A)}$}{$P(A|B) = \frac{P(A \cap B)}{P(A)}$}{$P(A|B) = \frac{P(A \cap B)}{P(B)}$}{$P(A|B) = \frac{P(B|A)}{P(B|A)}$}{a}

\textbf{Answer the next THREE questions based on the following information}

\begin{table}[h]
\centering
\begin{tabular}{cccc}
X & 0 & 1 & 2 \\ \hline
P(x) & $\frac 13$ & $\frac14$ & $\frac5{12}$
\end{tabular}
\end{table}

\question \textbf{What is the value of $E(X)$}
\choice{$\frac{15}{12}$}{$\frac{13}{12}$}{$\frac{1}{12}$}{$\frac{11}{13}$}{b}

\question \textbf{What is the value of $E(X^2)$}
\choice{$\frac{25}{12}$}{$\frac{13}{12}$}{$\frac{23}{12}$}{$\frac{25}{13}$}{c}

\question \textbf{What is $V(2X)$?}
\choice{2.93}{2.91}{1.97}{2.97}{d}

\question \textbf{10 out of each 100 people in a city walk to the office. If one is picked randomly, what is the probability s/he does not walk to the office?}
\choice{0.95}{0.10}{0.90}{0.01}{c}

\question \textbf{The third axiom of probability is --}
\choice{$0 \le P(A) \le 1$}{$P(S) = 1$}{$\displaystyle P(A_1 U A_2 U \cdots U
A_n) = \sum_{i=1}^{\infty}P(A_i)$}{$P(A) = 1 - P(A)$}{c}

\textbf{Answer the next three questions using the following information}

$P(A) = \frac 1 3, P(B) = \frac 1 2 \space \& \space P(A\cup B) = \frac 7 {12}$

\question \textbf{$P(A \cap B) = ?$}
\choice{$\frac {5}{12}$}{$\frac12$}{$\frac{1}{4}$}{$\frac{15}{16}$}{c}

\question \textbf{$P(A \cap \bar B)=?$}
\choice{$\frac{1}{4}$}{$\frac{3}{4}$}{$\frac{5}{6}$}{$\frac{1}{12}$}{a}

\question \textbf{What is the probability that B occurs or A does not occur?}
\choice{$\frac{3}{4}$}{$\frac{7}{12}$}{$\frac{5}{12}$}{$\frac{11}{12}$}{d}


\question \textbf{Possible value of probability}

i. -1 \quad
ii. 0.5 \quad
iii. 0

\textbf{Which one is correct?}

\choice{i and ii}{i and iii}{ii and iii}{i, ii and iii}{c}

\question \textbf{A set of sample points tabulated along with their respective probabilities is an example of -- }
\choice{Probability distribution}{Probability function}{Frequency distribution}{Marginal probability distribution}{a}

\question \textbf{Which one is a property of marginal probability density function?}
\choice{$\displaystyle \int_{x} f(x^2) \,dx=1$}{$\displaystyle  \int_{x} f(x^2) \,dx=0.5$}{$\displaystyle  \int_{x} f(x) \,dx=1$}{$P(x \ge 1)$}{c}

\question \textbf{Integrated value of $\frac 14 x^4$ --}
\choice{$\frac 1{20} x^5$}{$\frac 1{20} x^5+c$}{$\frac 1{5} x^4$}{$\frac 5{4} x^5$}{b}

\question \textbf{Which one is NOT an example of a continuous random variable -- }
\choice{Weight}{Height}{Time}{Size of television}{d}



\textbf{Answer the next THREE questions using the following information}

\begin{center}
$\displaystyle P(x) = \frac{x+1}{k}; x = 1,2,3,4$
\end{center}

\question \textbf{What is the value of k?}
\choice{10}{11}{14}{15}{c}

\question \textbf{$F(2)=-$}
\choice{$\frac{2}{14}$}{$\frac{3}{11}$}{$\frac{5}{14}$}{$\frac{5}{11}$}{c}

\question \textbf{$P(x)$ is a --}
\choice{Joint probability distribution}{Cumulative probability distribution}{Probability mass function}{Probability Density function}{c}

\question \textbf{A coin is tossed twice and no. of heads appeared is denoted by X. How many possible values of X are there?}
\choice{1}{2}{0}{3}{d}

\textbf{Answer the next two questions based on the following information}

\begin{table}[h]
\centering
\begin{tabular}{cccc}
X & 0 & 1 & 2 \\ \hline
P(x) & $\frac 12$ & $\frac14$ & $\frac14$
\end{tabular}
\end{table}

\question \textbf{What is F(1)}
\choice{$0.65$}{$0.75$}{$0.5$}{$1$}{b}

\question \textbf{$P(X \le 1 \le 3) =$--}
\choice{0.75}{0.70}{0.95}{1}{a}

\question \textbf{If $E(X) = -0.5$, then $E(1-2X) = $?}
\choice{0}{-1}{2}{1}{c}

%\question \textbf{To complete the song, the last answer should be
%\choice{a}{b}{c}{d}{e} % Invalid answer choice

\end{questions}

\pagebreak
%\newpage  %Uncomment to put on new age
\bigskip

\begin{multicols}{3}
[
Answer Key
]
\showallanswers % Phil Hirschorn
\end{multicols}


\end{document}