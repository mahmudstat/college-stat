\documentclass{article}
\usepackage{geometry}
\usepackage{amsfonts}

\geometry{
legalpaper, total={177.8mm, 290mm},left=20mm,
top=27mm, bottom=27mm,
}

\begin{document}

\begin{table}[h]
\centering
\begin{tabular}{lllll}
\textbf{\large SYLHET CADET COLLEGE} &  &  &  &  \\ \cline{4-5} 
PRE-TEST EXAMINATION - 2023 &  & \multicolumn{1}{l|}{} & \multicolumn{1}{l|}{Set} & \multicolumn{1}{l|}{D} \\ \cline{4-5} 
CLASS: XII &  &  &  &  \\ \cline{3-5} 
STATISTICS (CREATIVE)& \multicolumn{1}{l|}{\textbf{Subject Code:}} & \multicolumn{1}{l|}{1} & \multicolumn{1}{l|}{2} & \multicolumn{1}{l|}{9} \\ \cline{3-5} 
 SECOND PAPER &  &  &  &  \\
TIME – 2 hours \& 35 minutes &  &  &  &  \\
FULL MARKS – 50 &  &  &  & 
\end{tabular}
\end{table}
%  \normalfont\normalsize
 % 11.45a.m.~--~1.45p.m.

\hrule

\begin{center}
[\textbf{N.B.} – The figures of the right margin indicate full marks. Read the stems carefully and answer the associated questions. Answer any \textbf{FIVE} questions taking at least two from each group.]\\

\end{center}

  \begin{center}
  \textbf{Group--A}
  \end{center}
  
  \begin{enumerate}
  
 \item
	  \textbf{A quality control analyst in an industry tracks the no. of defective items produced per day. He observes 150 successive days and then prepares a table.} 
	  
	  \begin{table}[h]
	  \centering
\begin{tabular}{c|c|c|c|c|c} \hline
No. of items & 0 & 1 & 2 & 3 & 4 \\ \hline
Frequency & 30 & 32 & 40 & 28 & 20 \\ \hline
\end{tabular}
\end{table}
  
  \begin{enumerate}
    \item
	What is the formula of classical probability? \hfill 1
    \item
	Explain the difference between Priori Approach and Empirical Approach of probability. \hfill 2
    \item  
	What is the probability that less than 2 defective items would be produced on a particular day? \hfill 3
    \item
	Explain the relationship between independency and mutual excluvity in the light of the stem. \hfill 4
  \end{enumerate}
  
    \begin{center}
  \textbf{Group--B}
  \end{center}
  


\begin{center}
\textbf{\textit{“It is a capital mistake to theorize before one has data.”} – Sir Arthur Conan Doyle}
\end{center}
  
\end{enumerate}
\end{document}