\documentclass{exam}
%\documentclass[11pt,a4paper]{exam}
\usepackage{amsmath,amsthm,amsfonts,amssymb,dsfont}
\usepackage{ifthen}
\usepackage[legalpaper, total={7.00in, 11.31in}]{geometry}
\usepackage{enumerate}% http://ctan.org/pkg/enumerate
\usepackage{multicol}



% Accumulate the answers. Unmodified from Phil Hirschorn's answer
% https://tex.stackexchange.com/questions/15350/showing-solutions-of-the-questions-separately/15353
\newbox\allanswers
\setbox\allanswers=\vbox{}

\newenvironment{answer}
{%
    \global\setbox\allanswers=\vbox\bgroup
    \unvbox\allanswers
}%
{%
    \bigbreak
    \egroup
}

\newcommand{\showallanswers}{\par\unvbox\allanswers}
% End Phil's answer


% Is there a better way?
\newcommand*{\getanswer}[5]{%
    \ifthenelse{\equal{#5}{a}}
    {\begin{answer}\thequestion. (a)~#1\end{answer}}
    {\ifthenelse{\equal{#5}{b}}
        {\begin{answer}\thequestion. (b)~#2\end{answer}}
        {\ifthenelse{\equal{#5}{c}}
            {\begin{answer}\thequestion. (c)~#3\end{answer}}
            {\ifthenelse{\equal{#5}{d}}
                {\begin{answer}\thequestion. (d)~#4\end{answer}}
                {\begin{answer}\textbf{\thequestion. (#5)~Invalid answer choice.}\end{answer}}}}}
}

\setlength\parindent{0pt}
%usage \choice{ }{ }{ }{ }
%(A)(B)(C)(D)
\newcommand{\fourch}[5]{
    \par
    \begin{tabular}{*{4}{@{}p{0.23\textwidth}}}
        (a)~#1 & (b)~#2 & (c)~#3 & (d)~#4
    \end{tabular}
    \getanswer{#1}{#2}{#3}{#4}{#5}
}

%(A)(B)
%(C)(D)
\newcommand{\twoch}[5]{
    \par
    \begin{tabular}{*{2}{@{}p{0.46\textwidth}}}
        (a)~#1 & (b)~#2
    \end{tabular}
    \par
    \begin{tabular}{*{2}{@{}p{0.46\textwidth}}}
        (c)~#3 & (d)~#4
    \end{tabular}
    \getanswer{#1}{#2}{#3}{#4}{#5}
}

%(A)
%(B)
%(C)
%(D)
\newcommand{\onech}[5]{
    \par
    (a)~#1 \par (b)~#2 \par (c)~#3 \par (d)~#4
    \getanswer{#1}{#2}{#3}{#4}{#5}
}

\newlength\widthcha
\newlength\widthchb
\newlength\widthchc
\newlength\widthchd
\newlength\widthch
\newlength\tabmaxwidth

\setlength\tabmaxwidth{0.96\textwidth}
\newlength\fourthtabwidth
\setlength\fourthtabwidth{0.25\textwidth}
\newlength\halftabwidth
\setlength\halftabwidth{0.5\textwidth}

\newcommand{\choice}[5]{%
\settowidth\widthcha{AM.#1}\setlength{\widthch}{\widthcha}%
\settowidth\widthchb{BM.#2}%
\ifdim\widthch<\widthchb\relax\setlength{\widthch}{\widthchb}\fi%
    \settowidth\widthchb{CM.#3}%
\ifdim\widthch<\widthchb\relax\setlength{\widthch}{\widthchb}\fi%
    \settowidth\widthchb{DM.#4}%
\ifdim\widthch<\widthchb\relax\setlength{\widthch}{\widthchb}\fi%

% These if statements were bypassing the \onech option.
% \ifdim\widthch<\fourthtabwidth
%     \fourch{#1}{#2}{#3}{#4}{#5}
% \else\ifdim\widthch<\halftabwidth
% \ifdim\widthch>\fourthtabwidth
%     \twoch{#1}{#2}{#3}{#4}{#5}
% \else
%      \onech{#1}{#2}{#3}{#4}{#5}
%  \fi\fi\fi}

% Allows for the \onech option.
\ifdim\widthch>\halftabwidth
    \onech{#1}{#2}{#3}{#4}{#5}
\else\ifdim\widthch<\halftabwidth
\ifdim\widthch>\fourthtabwidth
    \twoch{#1}{#2}{#3}{#4}{#5}
\else
    \fourch{#1}{#2}{#3}{#4}{#5}
\fi\fi\fi}


\begin{document}

\begin{center}
  \bfseries\large
  Sylhet Cadet College

\normalsize
  Model Test Examination - 2023

  Class: HSC

  Subject: Statistics Second Paper (MCQ)

  Time: 25 minutes \qquad \qquad \qquad \qquad Subject Code: 130   \qquad \qquad \qquad  \qquad Full Marks: 25

%  \normalfont\normalsize
 % 11.45a.m.~--~1.45p.m.
\end{center}

\textbf{Answer all the questions. Each question is worth one (1) mark.}  

\begin{questions}

\question \textbf{Tossing a coin twice generates how many outcomes?}
\choice{4}{16}{8}{2}{a}

\textbf{Answer the next three questions based on the following information.}

A card is drawn from of pack of playing cards.

\question \textbf{What is the probability that the card is a King?}
\choice{0.0192}{0.25}{0.5}{0.0769}{d}

\question \textbf{P(The card is not from Diamonds)--}
\choice{$\frac12$}{$0$}{$\frac34$}{$\frac14$}{c}

\question \textbf{P(The card is red or Clubs)}
\choice{$\frac14$}{$\frac12$}{$\frac23$}{$\frac34$}{d}

\question \textbf{If a neutral die is thrown, the probability of having a digit greater than 6 is}
\choice{$\frac 1 6$}{$\frac 0 6$}{$\frac 2 3$}{$\frac 3 6$}{b}

\question \textbf{Tossing a coin twice generates how many outcomes?}
\choice{4}{16}{8}{2}{a}

\question \textbf{How many conditions does a probability density function have?}
\choice{2}{3}{4}{5}{b}

\question \textbf{The conditions of a probability distribution are--}

i. $\sum P(X) = 1$

ii. $\sum P(X) = 0$

iii. $0 \le P(X) \le 1$

\choice{i and ii}{i and iii}{ii and iii}{i, ii and iii}{b}

\textbf{Answer the next two questions using the following information}

\begin{table}[h]
	    \centering
\begin{tabular}{ccccccl}
x    & 1 & 2  & 3  & 4  & 5  & 6  \\ \hline
P(x) & k & 2k & 3k & 4k & 5k & 6k
\end{tabular}
\end{table}

\question \textbf{What is the value of k?}
\choice{$\frac{7}{21}$}{$\frac{5}{21}$}{$\frac{1}{21}$}{$1$}{c}

\question \textbf{What is the type of variable X?}
\choice{Discrete}{Discrete random}{Continuous}{Continuous random}{b}

\question \textbf{What is $F(\infty)$ for a distribution function $F(x)$?}
\choice{$-\infty$}{-1}{0}{1}{d}

\question \textbf{The example of a discrete random variable is--}

i. Binomial variate 

ii. Poisson variate

iii. Normal variate 

\textbf{Which one is correct?}

\choice{i and ii}{i and iii}{ii and iii}{i, ii and iii}{a} 

\question \textbf{What is the expected value of of the squared deviation of the value of the random variable from their mean?}
\choice{Arithmetic Mean}{Expectation}{Variance}{Co-variance}{c}

\question \textbf{Expected value of a constant a is --}
\choice{1}{Variance}{a}{a+1}{c}

\question \textbf{What is the minimum value of variance a random variable?}
\choice{$-\infty$}{1}{0}{-1}{c}

\question \textbf{If $y=ax+b$, what is the value of $E(y)?$}
\choice{$aE(X) + b$}{$a^2E(X)$}{$E(X)$}{$b$}{a}

\question \textbf{What is the value of $V(5)$?}
\choice{0}{25}{5}{1}{a}

\question \textbf{If $P(x) = \frac 1n; x = 1,2,3,\cdots ,n$, what is the value of $E(X)?$}
\choice{$\frac n2$}{$\frac{n-1}{2}$}{$\frac{n+1}{2}$}{$n+1$}{c}

\question \textbf{If $P(x)= \frac{4-|5-x|}{k}; x=2,3,4, \cdots 8$, what is the value of k?}
\choice{5}{8}{16}{24}{c}

\question \textbf{How many parameters are there in a binomial distribution?}
\choice{1}{2}{3}{4}{b}

\question \textbf{In a Binomial distribution, how are mean and variance related?}
\choice{$Mean > Variance$}{$Mean < Variance$}{$Mean = Variance$}{$Mean =2 \times Variance$}{a}

\question \textbf{When does Binomial distribution tend to Poisson distribution?}
\choice{$n \rightarrow \infty$ and $p \rightarrow \infty$}{$n \rightarrow 0$ and $p \rightarrow 0$}{$n \rightarrow \infty$ and $p \rightarrow 0$}{$n \rightarrow 0$ and $p \rightarrow \infty$}{c}

\question \textbf{What is the mean of Poisson distribution?}
\choice{$\frac 1{\sqrt m}$}{$m$}{$\frac 1m$}{$1+\frac 1m$}{b}

\question \textbf{The parameter of a Poisson variate is 2. What is its variance?}
\choice{0}{4}{$\sqrt 2$}{2}{d}

\question \textbf{The number of people living per unit area is called--}
\choice{Population Index}{Population Density}{Human Development Index}{Dependency Ratio}{b}

\question \textbf{Which formula of GFR is accurate?}
\choice{$GFR = \frac{B}{P}\times 1000$}{$GFR = \frac{B}{F_{15-49}}\times 1000$}{$GFR = \frac{B_i}{F_i}\times 1000$}{$GFR = \frac{G_i}{F{15-49}}\times 1000$}{b}

%\question \textbf{To complete the song, the last answer should be
%\choice{a}{b}{c}{d}{e} % Invalid answer choice

\end{questions}

\newpage  %Uncomment to put on new age
\bigskip

\begin{multicols}{3}
[
Answer Key
]
\showallanswers % Phil Hirschorn
\end{multicols}


\end{document}