\documentclass{article}
\usepackage{geometry}
\usepackage{amsfonts}

\geometry{
a4paper, total={170mm, 257mm},left=20mm,
top=20mm,
}

\begin{document}

\begin{center}
  \bfseries\large
  Sylhet Cadet College

\normalsize
  Model Test Examination - 2023

  Class: HSC

  Subject: Statistics Second Paper (Creative)

  Time: 2 hour \& 35 minutes \qquad \qquad \qquad Subject Code: 130  \qquad  \qquad \qquad Full Marks: 50

%  \normalfont\normalsize
 % 11.45a.m.~--~1.45p.m.
\end{center}

\noindent
\begin{tabular}{p{\dimexpr\linewidth-2\tabcolsep}}
  Answer FIVE questions taking at least two (2) from each group. Figures in the right indicate full marks.\\
  \hline
\end{tabular}

\begin{center}
\textbf{Group A}
\end{center}

  \begin{enumerate}

 \item
	  \textbf{A box contains four blue and 6 green balls. 3 balls are drawn randomly.} 
  
  \begin{enumerate}
    \item
	What is the value of $^nC_r$? \hfill 1
    \item
	Illustrate the difference between permutation and combination with an example. \hfill 2
    \item  
	What is the probability that all balls are green? \hfill 3
    \item
	What is the probabilith that one ball has a different color? \hfill 4
  \end{enumerate}
  
   \item
  
  	  \textbf{A red and a blue dice are thrown once. The dice are absolutely neutral and independent.} 
  
  \begin{enumerate}
    \item
	What is a simple event? \hfill 1
    \item
	Give an example of a certain event using set theory. \hfill 2
    \item  
	Find the probability that the difference of two digits from two dices is less than 3.\hfill 3
    \item
	Are the probabilities of getting greater digit from the blue die and that from the red die equal? Justify. \hfill 4
  \end{enumerate}
  
      \item
  \textbf{The joint probability function of two random variables X and Y is given below:}
  
  $\displaystyle P(X,Y) = \frac {x+2y}{16}; x = 0, 1; y = 0 ,1,2,3$
 
  \begin{enumerate}
    \item
	Write down the formula of conditional proibability. \hfill 1
    \item
    	What is the relationship between marginal and joint probability? \hfill 2
    \item
    	Find P(X). \hfill 3
     \item
     	Find $P(X\vert Y)$ and $P(X\vert 0)$. \hfill 4
  \end{enumerate}
  
     \item
	  \textbf{A box contains 5 red and 6 white balls. 3 balls are drawn at random. X is the number of white balls drawn.} 
  
  \begin{enumerate}
    \item
	What does variance measure? \hfill 1
    \item
	Can the variance be smaller than standard deviation? \hfill 2
    \item  
	Find the E(X) from the stem. \hfill 3
    \item
	Find the variance from the stem assuming X is the number of red balls drawn. \hfill 4
  \end{enumerate}

\begin{center}
\textbf{Group B}
\end{center}

   \item
  \textbf{The probability density function of a continuous random variable is}

$ \displaystyle f(x) = k(x+1); 0 \le x \le 1$

  \begin{enumerate}
    \item
	What is a random variable? \hfill 1
    \item
    	Find the value of k \hfill 2
    \item
    	Find the probability that the values of x would lie between 0 and 0.5. \hfill 3
     \item
     	What is the probability that X is greater than 0.8?  \hfill 4
  \end{enumerate}
  
   \item
	  \textbf{A farmer plans to store rice seeds for future use. It was found that 8 out of 20 seeds are rotten. He then collected a sample of 15 seeds.} 
  
  \begin{enumerate}
    \item
	What is Bernoulli trial? \hfill 1
    \item
	How are Bernoulli and Binomial distributions related? \hfill 2
    \item  
	What is the probability that at least one seed is rotten out of 15? \hfill 3
    \item
	What is the probability that the number of rotten seeds is greater than the arithmetic mean? \hfill 4
  \end{enumerate}
  
  \item
  \textbf{In winter, the probability that it rains on a particular day is 0.015. An analyst observes \\ 100 winter days.}
 
  \begin{enumerate}
    \item
	What is an experiment? \hfill 1
    \item
    	When can the Poisson distribution be approximated by the Binomial distribution? \hfill 2
    \item
    	Find, using Binomial distribution, the probability that  it would not rain at all on the \\ observed days. \hfill 3
     \item
     	Find the probability in 3(c) using Poisson distribution.  \hfill 4
  \end{enumerate}
  
   \item
  \textbf{For projection of population in a future time period, demographers use simple, \\ geometric or exponential growth technique. Each method has its advantages and \\ disadvantages.}

  \begin{enumerate}
    \item
	What is geometric growth? \hfill 1
    \item
    	In geometric growth method, obtain the formula for time required for the population to get \\ doubled [denote rate as r]. \hfill 2
    \item
    	In exponential method, how much unit of time is required for the population to get tripled?  \hfill 3
     \item
     	For projecting (predicting future values), is geometric growth method better than the \\ exponential method? Justify.  \hfill 4
  \end{enumerate}

\end{enumerate}

\end{document}