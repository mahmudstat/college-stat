\documentclass{article}
\usepackage{geometry}
\usepackage{amsfonts}

\geometry{
a4paper, total={170mm, 257mm},left=20mm,
top=20mm,
}

\begin{document}

\begin{center}
  \bfseries\large
  Sylhet Cadet College

\normalsize
  Model Test Examination - 2023

  Class: HSC

  Subject: Statistics First Paper (Creative)

  Time: 2 hour \& 35 minutes \qquad \qquad \qquad Subject Code: 129  \qquad  \qquad \qquad Full Marks: 50

%  \normalfont\normalsize
 % 11.45a.m.~--~1.45p.m.
\end{center}

\noindent
\begin{tabular}{p{\dimexpr\linewidth-2\tabcolsep}}
  Answer FIVE questions taking at least 2 (two) from each group. Figures in the right indicate full marks.\\
  \hline
\end{tabular}

\begin{center}
\textbf{Group A}
\end{center}

  \begin{enumerate}
 \item 
	  \textbf{Marks of six studnets in Physics and Chemsitry are as follow:} 
	  
	  Physics (X)   : 10, 14, 15, 13, 17, 18
	  Chemistry (Y) : 18, 16, 13, 10, 9, 14
  
  \begin{enumerate}
    \item
	Who is the Father of statistics? \hfill 1
    \item
	Write the steps in a statistical analysis. \hfill 2
    \item  
	Find the largest value: $\displaystyle \sum x_i^2, \sum y_i^2, (\sum x_i)^2$ \hfill 3
    \item
	Which subject seems easier? Show mathematically using proper notations. \hfill 4
  \end{enumerate}
  
   \item
	  \textbf{A sports analyst collected ages of athelets having ages between 10 and 35. He then presented his findings as below:} 
	  
	  \begin{table}[h]
	    \centering
\begin{tabular}{c|c|c|c|c|c}
Age            & 10-15 & 15-20 & 20-25 & 25-30 & 30-35 \\ \hline
No. of Athlete & 2     & 8     & 10    & 5     & 3    
\end{tabular}
\end{table}
  
  \begin{enumerate}
    \item
	What is central tendency? \hfill 1
    \item
	When is geometric mean iappropriate to measure? \hfill 2
    \item  
	Compute median from the stem. \hfill 3
    \item
	Show that Arithmetic mean is greater than Harmonic mean. Which one of them is more suitable for this data? \hfill 4
  \end{enumerate}
  
   \item
  \textbf{Frequency distribution of marks in statistics of a college is given in the following table.}
 

\begin{table}[h]
\centering
\begin{tabular}{ccc}
\hline
Marks & \begin{tabular}[c]{@{}c@{}}Number of Students\\ Group - A\end{tabular} & \begin{tabular}[c]{@{}c@{}}Number of Students\\ Group - B\end{tabular} \\ \hline
25-30 & 11 & 10 \\ 
30-35 & 18 & 16 \\ 
35-40 & 21 & 22 \\ 
40-45 & 26 & 28 \\ 
45-50 & 14 & 9 \\ \hline
\end{tabular}
\end{table}

  \begin{enumerate}
    \item
	What is data? \hfill 1
    \item
	What are the disadvantages of secondary data? \hfill 2
    \item  
	Calculate the arithmetic mean of Group - A \hfill 3
    \item
	Compute the combined mean. Is it greather than the arithmetic mean of Group - B? Explain the possible reason(s). \hfill 4
\end{enumerate}

   \item
	  \textbf{The arithmetic and geometric means of the first and third quartiles of a distribution are 10 and 8, respectively. The second quartile is 10.} 
  
  \begin{enumerate}
    \item
	What is the formula suggested by Pearson to find skewness? \hfill 1
    \item
	Which moments are useful in measuring central tendency and dispersion?  \hfill 2
    \item  
	Find skewness from the stem using a suitable formula. \hfill 3
    \item
	Which method of finding skewness od you think is the best and why? \hfill 4
\end{enumerate}
  
  \newpage

\begin{center}
\textbf{Group B}
\end{center}
  
 \item
	  \textbf{The first four moments about 3 of a distribution are -1, 5, -10, and 120.} 
  
  \begin{enumerate}
    \item
	What are moments used for? \hfill 1
    \item
	Can the second central moment be greater than the third central moment? \hfill 2
    \item  
	Find the second and third moments about arithmetic mean of the distribution. \hfill 3
    \item
	Find skewness and kurtosis and comment on the values.  \hfill 4
\end{enumerate}
  
 \item
	  \textbf{A clyclist moves around a square-shaped lake with the speeds 20, 25, 30, and 16 km per hour.} 
  
  \begin{enumerate}
    \item
	What is grouped data? \hfill 1
    \item
	Is arithmetic mean suitable for this data? \hfill 2
    \item  
	Find the average speed of the cyclist. \hfill 3
    \item
	Can we use some other formula for finding the average? Demonstrate. \hfill 4
  \end{enumerate}
  
 \item
	  \textbf{Income of a freelancer in 6 successive months (from Jan to Jun) was found to be \\ 46.0, 49.5, 51.5, 50.6, 56.5, and 60 (in thousands BDT.).}
  \begin{enumerate}
    \item
	What is time series data? \hfill 1
    \item
	What are the components of a time series model? \hfill 2
    \item  
	Determine the 3-monthly moving average from the data. \hfill 3
    \item
	Draw the moving averages on a graph paper and interpret. \hfill 4
\end{enumerate}

 \item
	  \textbf{In 2015, tens of thousands of Rohingya people were forcibly displaced from their villages and IDP camps in Rakhine state, Mynmar. Many of them fled to neighboring countries, including Bangladesh, Malaysia, Indonesia. Many national and internations agencies collect data on the issue.} 
  
  \begin{enumerate}
    \item
	What is non-official statistics? \hfill 1
    \item
	Name five sources of official statistics. \hfill 2
    \item  
	Shed some light on the limitations of official statistics. \hfill 3
    \item
	How can the quality of published statistics in Bangladesh e improved? \hfill 4
  \end{enumerate}

\end{enumerate}

\end{document}