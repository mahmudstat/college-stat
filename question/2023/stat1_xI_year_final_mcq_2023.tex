\documentclass{exam}
%\documentclass[11pt,a4paper]{exam}
\usepackage{amsmath,amsthm,amsfonts,amssymb,dsfont}
\usepackage{ifthen}
\usepackage[legalpaper, total={177.8mm, 290mm}]{geometry}
\usepackage{enumerate}% http://ctan.org/pkg/enumerate
\usepackage{multicol}



% Accumulate the answers. Unmodified from Phil Hirschorn's answer
% https://tex.stackexchange.com/questions/15350/showing-solutions-of-the-questions-separately/15353
\newbox\allanswers
\setbox\allanswers=\vbox{}

\newenvironment{answer}
{%
    \global\setbox\allanswers=\vbox\bgroup
    \unvbox\allanswers
}%
{%
    \bigbreak
    \egroup
}

\newcommand{\showallanswers}{\par\unvbox\allanswers}
% End Phil's answer


% Is there a better way?
\newcommand*{\getanswer}[5]{%
    \ifthenelse{\equal{#5}{a}}
    {\begin{answer}\thequestion. (a)~#1\end{answer}}
    {\ifthenelse{\equal{#5}{b}}
        {\begin{answer}\thequestion. (b)~#2\end{answer}}
        {\ifthenelse{\equal{#5}{c}}
            {\begin{answer}\thequestion. (c)~#3\end{answer}}
            {\ifthenelse{\equal{#5}{d}}
                {\begin{answer}\thequestion. (d)~#4\end{answer}}
                {\begin{answer}\textbf{\thequestion. (#5)~Invalid answer choice.}\end{answer}}}}}
}

\setlength\parindent{0pt}
%usage \choice{ }{ }{ }{ }
%(A)(B)(C)(D)
\newcommand{\fourch}[5]{
    \par
    \begin{tabular}{*{4}{@{}p{0.23\textwidth}}}
        (a)~#1 & (b)~#2 & (c)~#3 & (d)~#4
    \end{tabular}
    \getanswer{#1}{#2}{#3}{#4}{#5}
}

%(A)(B)
%(C)(D)
\newcommand{\twoch}[5]{
    \par
    \begin{tabular}{*{2}{@{}p{0.46\textwidth}}}
        (a)~#1 & (b)~#2
    \end{tabular}
    \par
    \begin{tabular}{*{2}{@{}p{0.46\textwidth}}}
        (c)~#3 & (d)~#4
    \end{tabular}
    \getanswer{#1}{#2}{#3}{#4}{#5}
}

%(A)
%(B)
%(C)
%(D)
\newcommand{\onech}[5]{
    \par
    (a)~#1 \par (b)~#2 \par (c)~#3 \par (d)~#4
    \getanswer{#1}{#2}{#3}{#4}{#5}
}

\newlength\widthcha
\newlength\widthchb
\newlength\widthchc
\newlength\widthchd
\newlength\widthch
\newlength\tabmaxwidth

\setlength\tabmaxwidth{0.96\textwidth}
\newlength\fourthtabwidth
\setlength\fourthtabwidth{0.25\textwidth}
\newlength\halftabwidth
\setlength\halftabwidth{0.5\textwidth}

\newcommand{\choice}[5]{%
\settowidth\widthcha{AM.#1}\setlength{\widthch}{\widthcha}%
\settowidth\widthchb{BM.#2}%
\ifdim\widthch<\widthchb\relax\setlength{\widthch}{\widthchb}\fi%
    \settowidth\widthchb{CM.#3}%
\ifdim\widthch<\widthchb\relax\setlength{\widthch}{\widthchb}\fi%
    \settowidth\widthchb{DM.#4}%
\ifdim\widthch<\widthchb\relax\setlength{\widthch}{\widthchb}\fi%

% These if statements were bypassing the \onech option.
% \ifdim\widthch<\fourthtabwidth
%     \fourch{#1}{#2}{#3}{#4}{#5}
% \else\ifdim\widthch<\halftabwidth
% \ifdim\widthch>\fourthtabwidth
%     \twoch{#1}{#2}{#3}{#4}{#5}
% \else
%      \onech{#1}{#2}{#3}{#4}{#5}
%  \fi\fi\fi}

% Allows for the \onech option.
\ifdim\widthch>\halftabwidth
    \onech{#1}{#2}{#3}{#4}{#5}
\else\ifdim\widthch<\halftabwidth
\ifdim\widthch>\fourthtabwidth
    \twoch{#1}{#2}{#3}{#4}{#5}
\else
    \fourch{#1}{#2}{#3}{#4}{#5}
\fi\fi\fi}


\begin{document}

\begin{center}
  \bfseries\large
  Sylhet Cadet College

\normalsize
  Half-Yearly Examination - 2023

  Class: XI

  Subject: Statistics First Paper (MCQ)  \fbox{Set: C}

  Time: 20 minutes \qquad \qquad \qquad \qquad Subject Code: 129   \qquad \qquad \qquad  \qquad Full Marks: 25

%  \normalfont\normalsize
 % 11.45a.m.~--~1.45p.m.
\end{center}

\noindent
\begin{tabular}{p{\dimexpr\linewidth-2\tabcolsep}}
 Answer all the questions. Each question is worth one (1) mark.\\
  \hline
\end{tabular}

\begin{questions}

\question \textbf{If $\displaystyle \sum_{i=1}^{20} x_i^2=20$ and $\displaystyle \sum_{i=1}^{20} x_i=30$, what is the value of $\displaystyle \sum_{i=1}^{20} x_i^2 + \sum_{i=1}^{20} x_i + 100$?}
\choice{130}{200}{150}{2130}{c}

\question \textbf{A subset of a population is called--}
\choice{Constant}{Variable}{Sample}{Scale}{c}

\question \textbf{$If x_1=2, x_2=-3, x_3=7$, and $x_4=12, \displaystyle \sum_{i=1}^4 x_i^2=?$}
\choice{26}{106}{206}{216}{c}

\question \textbf{Which one falls in the category of interval scale?}
\choice {Temperature}{Speed}{Distance}{Film rating}{a}

\question \textbf{Which is a discrete variable?}
\choice{Weight}{Amount of rainfall}{Distance}{Grade in a subject}{d}

\question \textbf{Which one is product of square?}
\choice {$\prod x_i^2$}{$(\prod x_i)^2$}{$\sum x_i^2 \times \sum x$}{$\sum x_i^2$}{a}

\question \textbf{How many measure of central tendency are there?}
\choice{2}{3}{4}{5}{d}

\question \textbf{Which one is smallest?}
\choice{$\displaystyle \sum_{i=1}^n (X_i-Median)^2$}{$\displaystyle \sum_{i=1}^n (X_i-\bar X)^2$}{$\displaystyle \sum_{i=1}^n (X_i-\sigma)^2$}{$\displaystyle \sum_{i=1}^n (X_i-Mode)^2$}{a}

\question \textbf{Which measure of central tendency is suitable for qualitative variable?}
\choice{Arithmetic Mean}{Harmonic Mean}{Quadratic Mean}{Mode}{d}

\question{\textbf{Inappropriate for algebraic analysis--}}

i. Median \\
ii. Mode \\
iii. Geometric Mean

Which one is true?

\choice{i}{ii}{i \& ii}{ii \& iii}{c}

\textbf{Answer the next two questions based on the following information}

  \begin{table}[h]
\centering
\begin{tabular}{cccccc}
Accident     & 4 & 6 & 7 & 8 & 9\\ \hline
Frequency & 2   & 0    & 4     & 4     & 1   
\end{tabular}
\end{table}
\question \textbf{Fifth Decile is --}
\choice{0}{8}{7}{6}{c}

\question \textbf{Which of the following is mode?}
\choice{4}{8}{0}{7}{b}

\question \textbf{Which data set is suitable for Geometric Mean?}
\choice{$1,-1,2,4,6,7$}{$1,2,4,8,16,32$}{$0,1,2,3,4,6$}{$1,1,2,3,4,4,5$}{b}

\question \textbf{Which of the following may be used to determine mode?}
\choice{Histogram}{Frequency Curve}{Ogive}{Frequency Polygon}{a}

\question \textbf{Which relationship is correct?}
\choice{$AM \times GM = HM^2$}{$AM \times HM = GM^2$}{$AM \times HM = GM^3$}{$AM \div GM = HM^2$}{b}

\question \textbf{In case of positive skewness, which one is correct?}
\choice{$Mean>Median>Mode$}{$Mean<Median<Mode$}{$Mean=Median=Mode$}{$Mean>Median<Mode$}{a}

\question \textbf{The first raw moment about 3 is -5. What is the value of arithmetic mean?}
\choice{2}{-2}{0}{8}{b}

\question \textbf{For a symmetrical distribution, $\beta_1=$}
\choice{1}{-1}{0}{3}{c}

\question \textbf{For a mesokurtik distribution, $\beta_2 = --$}
\choice{0}{-3}{3}{1}{c}

\question \textbf{Moments can be--}

i. positive \\
ii. not negative \\
iii. positive or negative

\textbf{Which one is correct?}

\choice{i and ii}{i and iii}{ii and iii}{i, ii and iii}{b}

\question \textbf{First moment around a is equal to --}
\choice{1}{0}{-1}{$\bar x-a$}{d}


\question \textbf{Which component involves period more than one (01) year?}
\choice{Seasonal Variation}{Cyclic Variation}{Irregular Variation}{Random Variation}{b}

\question \textbf{Which measure is unit-free?}
\choice{Range}{Mean deviation}{Standard deviation}{Coefficient of variation}{d}

\textbf{Answer the next two questions based on the following table:}

  \begin{table}[h]
\centering
\begin{tabular}{ccccccc}
Year     & 2007 & 2008 & 2009 & 2010 & 2011 & 2012\\ \hline
Sales & 5   & 35    & 34     & 40     & 42  & 204 
\end{tabular}
\end{table}

\question \textbf{In Semi-Average method, what is the 2nd average?}
\choice{74}{24.67}{95.33}{28}{c}

\question \textbf{What is the last  value of 3-yearly moving average?}
\choice{93.55}{95.53}{95.33}{59.33}{c}

%\question \textbf{To complete the song, the last answer should be
%\choice{a}{b}{c}{d}{e} % Invalid answer choice

\end{questions}

 \vspace{2.5cm}

\begin{center}
“Maturity is the capacity to endure uncertainty”--John Finley
\end{center}
\pagebreak
%\newpage  %Uncomment to put on new age
\bigskip

\begin{multicols}{3}
[
Answer Key
]
\showallanswers % Phil Hirschorn
\end{multicols}


\end{document}