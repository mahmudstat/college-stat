\documentclass{article}
\usepackage{geometry}
\usepackage{amsfonts}



\geometry{
legalpaper, total={177.8mm, 290mm},left=20mm,
top=27mm, bottom=27mm,
}

\begin{document}

\begin{center}
  \bfseries\large
  Sylhet Cadet College

\normalsize
  Progress Test Examination - 2023

  Class: HSC

  Subject: Statistics First Paper (Creative)

  Time: 2 hour \& 35 minutes \qquad \qquad \qquad Subject Code: 129  \qquad  \qquad \qquad Full Marks: 50

%  \normalfont\normalsize
 % 11.45a.m.~--~1.45p.m.
\end{center}

\noindent
\begin{tabular}{p{\dimexpr\linewidth-2\tabcolsep}}
  Answer FIVE questions taking at least two (2) from each group. Figures in the right indicate full marks.\\
  \hline
\end{tabular}

\begin{center}
\textbf{Group A}
\end{center}
  \begin{enumerate}

 \item
	  \textbf{Height (in inches) of 10 cadets in a class are: 50, 60, 55, 65, 66, 70, 54, 64, 62, 72} 
	 
  \begin{enumerate}
    \item
	What is population in statistics? \hfill 1
    \item
	Is height discrete or continuous? \hfill 2
    \item  
	Find $\displaystyle \sum_{i=1}^{10} x_i^2$ \hfill 3
    \item
	Find the square of mean and mean of square. Are they equal? \hfill 4
  \end{enumerate}
 \item
	  \textbf{For two non-zero positive numbers, $GM=4\sqrt3$ and $HM=6$, where the quantities bear usual notations}  
  \begin{enumerate}
    \item
	When is Harmonic mean suitable? \hfill 1
    \item
	For two numbers, what is the relationship between AM, GM, and HM? \hfill 2
    \item  
	What is the Arithmetic mean? \hfill 3
    \item
	Determine the numbers. \hfill 4
  \end{enumerate}
  
     \item
	  \textbf{12 is deducted from each value of a variable and then divided by 3. The new arithmetic mean (AM) is found to be 4.} 
  
  \begin{enumerate}
    \item
	What is change of origin? \hfill 1
    \item
	Does AM depend on origin? Prove with an example. \hfill 2
    \item  
	From the stem, find the original AM. \hfill 3
    \item
	Does the origin or the scale have greater impact on AM in this example? \hfill 4
  \end{enumerate}
    \item
  \textbf{In the test examination, marks of 11 students in statistics are: 90, 92, 93, 49, 44, 88, 80, 58, 83, 71, 76.}
  \begin{enumerate}
    \item
	What is central tendency? \hfill 1
    \item
	When is median better than arithmetic mean? Explain with an example. \hfill 2
    \item  
	Find the 3rd the quartile and 61st percentile from the data and explain.  \hfill 3
    \item
	Do quantiles depend on change of origin and scale. Prove using two examples.\hfill 4
\end{enumerate}

\begin{center}
\textbf{Group B}
\end{center}

   \item
	  \textbf{Exam marks of two students were summarized for the purpose of comnparison. The summary is given below:} 
	  
	  \begin{table}[h]
	  \centering
\begin{tabular}{c|c|c}
Measure         & Student X & Student Y \\ \hline
First Quartile  & 28        & 27        \\
Second Quartile & 60        & 60        \\
Third Quartile  & 75        & 73        \\
Minimum         & 16        & 14        \\ 
Maximum         & 89        & 86        \\  \hline
\end{tabular}
\end{table}
  
  \begin{enumerate}
    \item
	What is kurtosis? \hfill 1
    \item
	How much data are contained within Interquartile range? \hfill 2
    \item  
	For student A, estimate the Bowley's Coefficient of skewness and explain. \hfill 3
    \item
	On the basis of skewness (and hence shape of the data), compare the students. \hfill 4
  \end{enumerate}
  
       \item
	  \textbf{The yearly revenue (in hundred thousand) of shoe manufacturer company is given below} 
  \begin{table}[h]
\centering
\begin{tabular}{cccccccc}
Year     & 2005 & 2006 & 2007 & 2008 & 2009 & 2010 & 2011 \\ \hline
Revenue & 35   & 22    & 40     & 35     & 50     & 42 & 60   
\end{tabular}
\end{table}
  
  \begin{enumerate}
    \item
	What is general trend? \hfill 1
    \item
	Which method of determining trend gives only two values? \hfill 2
    \item  
	Determine the trend using three-yearly moving average method. \hfill 3
    \item
	Find the trend using graphical method and extrapolate the approximate revenue earned in 2012. \hfill 4
  \end{enumerate}
  
   \item
	  \textbf{Marks obtained by a student in 7 subjects are} 
	  \begin{center}
	  70, 66, 55, 45, 80, 30, 82
	\end{center}
  
  \begin{enumerate}
    \item
	What is negative skewness? \hfill 1
    \item
	Draw graphs of positive and negative skewness showing the locations of mean and median. \hfill 2
    \item  
	Determine the five number summary from the stem and explain. \hfill 3
    \item
	Are the data symmetric? If not, comment on the pattern of data. \hfill 4
\end{enumerate}
  

     \item
	  \textbf{Climate change is an alarming problem throughout the world. To determine what to do to solve the problem, many government and non-government organizatios collcet and analyze data  to come to a consistent solution.} 
  
  \begin{enumerate}
    \item
	What is official statistics?  \hfill 1
    \item
	What is the role of World Meteorological Organization? \hfill 2
    \item  
	What are the limitations of published statistics in Bangladesh? \hfill 3
    \item
	How can the quality of published statistics in Bangladesh be improved? \hfill 4
  \end{enumerate}
  
\end{enumerate}
\end{document}