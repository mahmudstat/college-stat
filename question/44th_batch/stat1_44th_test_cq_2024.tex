\documentclass{article}
\usepackage{geometry}
\usepackage{amsfonts}
\usepackage{float}

\geometry{
legalpaper, total={177.8mm, 290mm},left=20mm,
top=7mm, bottom=27mm,
}

\begin{document}

\begin{table}[h]
\centering
\begin{tabular}{lllll}
\textbf{\large SYLHET CADET COLLEGE} &  &  &  &  \\ \cline{4-5} 
TEST EXAMINATION - 2025 &  & \multicolumn{1}{l|}{} & \multicolumn{1}{l|}{Set:} & \multicolumn{1}{l|}{C} \\ \cline{4-5} 
CLASS: XII &  &  &  &  \\ \cline{3-5} 
STATISTICS (CREATIVE)& \multicolumn{1}{l|}{\textbf{Subject Code:}} & \multicolumn{1}{l|}{1} & \multicolumn{1}{l|}{2} & \multicolumn{1}{l|}{9} \\ \cline{3-5} 
 FIRST PAPER &  &  &  &  \\
TIME – 2 hrs \& 35 minutes &  &  &  &  \\
FULL MARKS – 50 &  &  &  & 
\end{tabular}
\end{table}
%  \normalfont\normalsize
 % 11.45a.m.~--~1.45p.m.

\hrule

\begin{center}
[\textbf{N.B.} – The figures of the right margin indicate full marks. Read the stems carefully and answer the associated questions. Answer any \textbf{FIVE} questions taking at least two questions from each group]\\


\end{center}
  \begin{enumerate}
  
  \item  
\textbf{The daily website visits (in thousands) for an online platform over seven days are recorded as 80, 85, 90, 95, 100, 105, and 110 (denoted by \textit{z}). The platform analyst claimed that the square of the total visits is greater than the total of the squared visits.}

\begin{enumerate}
    \item Give an example of an infinite population. \hfill 1
    \item Differentiate between discrete and continuous variable. \hfill 2
    \item  
    Calculate $\displaystyle \sum_{i=1}^7 (z_i - 2z_i)^2$ using the provided data. \hfill 3
    \item
    Verify whether the analyst’s statement is accurate based on the data. \hfill 4
\end{enumerate}

\item
\textbf{The number of books sold by a bookstore in 20 days is summarized as shown below.}

\begin{table}[h]
\centering
\begin{tabular}{|c|ccccc|}
Books Sold (X) & 0 & 1 & 2 & 3 & 4 \\ \hline
Days (Y) & 5 & 7 & 4 & 3 & 1
\end{tabular}
\end{table}

\begin{enumerate}
    \item What is bivariate data? \hfill 1
    \item Give one example of each scale of measurement. \hfill 2
    \item  
    Find the total number of books sold using a suitable notation. \hfill 3
    \item
    Verify the statement: $\displaystyle \sum_{i=1}^{5} X_iY_i = \sum_{i=1}^{5} X_i \times \sum_{i=1}^{5} Y_i$ \hfill 4
\end{enumerate}

\item
\textbf{For two positive non-zero numbers, if $GM = 6\sqrt{3}$ and $AM = 10$, where the symbols represent their usual meanings:}

\begin{enumerate}
    \item what is the relationship among AM, GM, and HM? \hfill 1
    \item Find Arithmetic Mean: $14,18,22, \cdots, 70$ \hfill 2
    \item  
    Determine the Harmonic Mean (HM) of the two numbers. \hfill 3
    \item
    Find the values of the two numbers. \hfill 4
    
\end{enumerate}

   \item
\textbf{The number of hours spent studying per week by students in a school were recorded as follows:}

\begin{table}[h]
\centering
\begin{tabular}{c|c}
\textbf{Hours Studied} & \textbf{Frequency} \\ \hline
0-5                   & 8                  \\ \hline
5-10                  & 12                 \\ \hline
10-15                 & 10                 \\ \hline
15-20                 & 6                  \\ \hline
20-25                 & 4                 
\end{tabular}
\end{table}

    \begin{enumerate}
    \item What is change of origin? \hfill 1
    \item
	Relate short-cut method of arithmetic mean with change of origin and scale. \hfill 2
    \item  
	Compute the Arithmetic Mean of the given data using the short-cut method. \hfill 3
    \item
	Compute the Arithmetic Mean with a different value of origin (a). Do both the methods give same result? What is the best choice of a? \hfill 4
  \end{enumerate}

% Questions here

\begin{center}
\textbf{Group  - B}
\end{center}

\item
\textbf{For a given data set representing the monthly salaries (in thousands) of employees in a company, the following statistics are provided: Median = 60, Mode = 55, Standard Deviation = 5, and Coefficient of Variation (CV) = 8.3.}

\begin{enumerate}
    \item How many types of moments are there?  \hfill 1
    \item Derive the value of the first central moment. \hfill 2
    \item  
    Calculate the skewness using Pearson's method ($SK_P$). \hfill 3
    \item
    Does the value of ($SK_P$) accurately reflect the nature of the data based on the given statistics? Justify your answer. \hfill 4
\end{enumerate}

\item
\textbf{The annual sales (in million dollars) of a tech company over eight years are provided below:}

\begin{table}[H]
\centering
\begin{tabular}{ccccccccc}
Year     & 2014 & 2015 & 2016 & 2017 & 2018 & 2019 & 2020 & 2021 \\ \hline
Sales    & 120  & 150  & 140  & 160  & 180  & 200  & 220  & 240   
\end{tabular}
\end{table}

\begin{enumerate}
 \item Write down the additive model of time series. \hfill 1
 \item How does semi-average method work? \hfill 2
    \item  
    Calculate the trend using the three-yearly moving average method. \hfill 3
    \item
    Predict the approximate sales for the year 2022 using both graphical and moving average methods. \hfill 4
\end{enumerate}

\item
\textbf{The weights of newborn babies (in kilograms) in a hospital over a week were recorded by a pediatrician to monitor their health. The weights of 10 randomly selected babies are given below:}

\begin{center}
\textbf{3.2, 2.9, 3.5, 3.1, 3.4, 3.0, 3.3, 3.6, 2.8, 3.7}
\end{center}

\begin{enumerate}
    \item Draw a symmetrical distribution.   \hfill 1
    \item Write three uses of five number summary \hfill 2
    \item  
    Represent the data using a Box \& Whisker plot. \hfill 3
    \item
    Compare the plot with the five-number summary and comment on the distribution of the data. \hfill 4
\end{enumerate}



  \item
\textbf{The role of official statistics in policy-making is crucial for economic and social development. Governments and international organizations rely on accurate data to formulate strategies and monitor progress.}

\begin{enumerate}
   \item What is semi-official statistics?  \hfill 1
 \item Briefly mention what Bangladesh bank does.  \hfill 2
    \item  
    Discuss the sources of official statistics in Bangladesh and their importance in decision-making. \hfill 3
    \item
    Analyze the limitations of official statistics in Bangladesh and suggest ways to improve their reliability. \hfill 4
\end{enumerate}


\vspace{3cm}

\begin{center}
 “The greatest value of a picture is when it forces us to notice what we never expected to see.” \\ \end{center}
\hfill – John Tukey.

  
\end{enumerate}
\end{document}