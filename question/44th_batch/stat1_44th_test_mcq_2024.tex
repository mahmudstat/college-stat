\documentclass{exam}
%\documentclass[11pt,a4paper]{exam}
\usepackage{amsmath,amsthm,amsfonts,amssymb,dsfont}
\usepackage{ifthen}
\usepackage{geometry}
\usepackage{float}
\geometry{
legalpaper, total={177.8mm, 290mm},left=20mm,
top=7mm, bottom=27mm,
}
\usepackage{enumerate}% http://ctan.org/pkg/enumerate
\usepackage{multicol}
\usepackage{hhline}
\usepackage[table]{xcolor}


% Accumulate the answers. Unmodified from Phil Hirschorn's answer
% https://tex.stackexchange.com/questions/15350/showing-solutions-of-the-questions-separately/15353
\newbox\allanswers
\setbox\allanswers=\vbox{}

\newenvironment{answer}
{%
    \global\setbox\allanswers=\vbox\bgroup
    \unvbox\allanswers
}%
{%
    \bigbreak
    \egroup
}

\newcommand{\showallanswers}{\par\unvbox\allanswers}
% End Phil's answer


% Is there a better way?
\newcommand*{\getanswer}[5]{%
    \ifthenelse{\equal{#5}{a}}
    {\begin{answer}\thequestion. (a)~#1\end{answer}}
    {\ifthenelse{\equal{#5}{b}}
        {\begin{answer}\thequestion. (b)~#2\end{answer}}
        {\ifthenelse{\equal{#5}{c}}
            {\begin{answer}\thequestion. (c)~#3\end{answer}}
            {\ifthenelse{\equal{#5}{d}}
                {\begin{answer}\thequestion. (d)~#4\end{answer}}
                {\begin{answer}\textbf{\thequestion. (#5)~Invalid answer choice.}\end{answer}}}}}
}

\setlength\parindent{0pt}
%usage \choice{ }{ }{ }{ }
%(A)(B)(C)(D)
\newcommand{\fourch}[5]{
    \par
    \begin{tabular}{*{4}{@{}p{0.23\textwidth}}}
        (a)~#1 & (b)~#2 & (c)~#3 & (d)~#4
    \end{tabular}
    \getanswer{#1}{#2}{#3}{#4}{#5}
}

%(A)(B)
%(C)(D)
\newcommand{\twoch}[5]{
    \par
    \begin{tabular}{*{2}{@{}p{0.46\textwidth}}}
        (a)~#1 & (b)~#2
    \end{tabular}
    \par
    \begin{tabular}{*{2}{@{}p{0.46\textwidth}}}
        (c)~#3 & (d)~#4
    \end{tabular}
    \getanswer{#1}{#2}{#3}{#4}{#5}
}

%(A)
%(B)
%(C)
%(D)
\newcommand{\onech}[5]{
    \par
    (a)~#1 \par (b)~#2 \par (c)~#3 \par (d)~#4
    \getanswer{#1}{#2}{#3}{#4}{#5}
}

\newlength\widthcha
\newlength\widthchb
\newlength\widthchc
\newlength\widthchd
\newlength\widthch
\newlength\tabmaxwidth

\setlength\tabmaxwidth{0.96\textwidth}
\newlength\fourthtabwidth
\setlength\fourthtabwidth{0.25\textwidth}
\newlength\halftabwidth
\setlength\halftabwidth{0.5\textwidth}

\newcommand{\choice}[5]{%
\settowidth\widthcha{AM.#1}\setlength{\widthch}{\widthcha}%
\settowidth\widthchb{BM.#2}%
\ifdim\widthch<\widthchb\relax\setlength{\widthch}{\widthchb}\fi%
    \settowidth\widthchb{CM.#3}%
\ifdim\widthch<\widthchb\relax\setlength{\widthch}{\widthchb}\fi%
    \settowidth\widthchb{DM.#4}%
\ifdim\widthch<\widthchb\relax\setlength{\widthch}{\widthchb}\fi%

% These if statements were bypassing the \onech option.
% \ifdim\widthch<\fourthtabwidth
%     \fourch{#1}{#2}{#3}{#4}{#5}
% \else\ifdim\widthch<\halftabwidth
% \ifdim\widthch>\fourthtabwidth
%     \twoch{#1}{#2}{#3}{#4}{#5}
% \else
%      \onech{#1}{#2}{#3}{#4}{#5}
%  \fi\fi\fi}

% Allows for the \onech option.
\ifdim\widthch>\halftabwidth
    \onech{#1}{#2}{#3}{#4}{#5}
\else\ifdim\widthch<\halftabwidth
\ifdim\widthch>\fourthtabwidth
    \twoch{#1}{#2}{#3}{#4}{#5}
\else
    \fourch{#1}{#2}{#3}{#4}{#5}
\fi\fi\fi}


\begin{document}

\begin{table}[h]
\centering
\begin{tabular}{lllll}
\textbf{\large SYLHET CADET COLLEGE} &  &  &  &  \\ \cline{4-5} 
TEST EXAMINATION - 2023 &  & \multicolumn{1}{l|}{} & \multicolumn{1}{l|}{Set} & \multicolumn{1}{l|}{D} \\ \cline{4-5} 
CLASS: XII &  &  &  &  \\ \cline{3-5} 
MULTIPLE CHOICE QUESTIONS & \multicolumn{1}{l|}{\textbf{Subject Code:}} & \multicolumn{1}{l|}{1} & \multicolumn{1}{l|}{2} & \multicolumn{1}{l|}{9} \\ \cline{3-5} 
STATISTICS FIRST PAPER &  &  &  &  \\
TIME – 25 minutes &  &  &  &  \\
FULL MARKS – 25 &  &  &  & 
\end{tabular}
\end{table}
%  \normalfont\normalsize
 % 11.45a.m.~--~1.45p.m.
\hrule

\begin{center}
[N.B. – Answer all the questions. Each question carries ONE mark. Block fully, with a black ball- point pen, the circle of the letter that stands for the correct/best answer in the “Answer sheet” for the Multiple Choice Questions Examination.]\\

  
  \textbf{Candidates are asked not to leave any mark or spot on the question paper.}
\end{center}
\begin{questions}


\question \textbf{For a symmetrical distribution, what is the value of $\beta_1?$}
\choice{0}{1}{-1}{$\infty$}{a}

\question \textbf{What is the formula of IQR?}
\choice{$IQR = Q_3 + Q_1$}{$IQR = Q_3 - Q_1$}{$IQR = 2Q_3 - Q_1$}{$IQR = \frac{Q_3 - Q_1}{2}$}{b}

\question \textbf{A survey categorizing people by their favorite color is an example of which measurement scale?}  
\choice{Nominal}{Ordinal}{Interval}{Ratio}{a}  

\question \textbf{If $\displaystyle \sum_{i=1}^{25} z_i^2=75$ and $\displaystyle \sum_{i=1}^{25} z_i=50$, compute $\displaystyle \sum_{i=1}^{25} z_i^2 + 2\sum_{i=1}^{25} z_i - 125$.}  
\choice{50}{75}{100}{25}{a} 

\question \textbf{If $y_1=5$, $y_2=2$, $y_3=-1$, and $y_4=4$, compute $\displaystyle \sum_{i=1}^4 (y_i^2 + 2)$.}  
\choice{50}{40}{44}{60}{c}  

\question \textbf{A good measure of central tendency -}  

i. is stable for different samples \\  
ii. provides a single representative value \\  
iii. ignores extreme values completely  

\textbf{Which one is correct?}  

\choice{i and ii}{i and iii}{ii and iii}{i, ii and iii}{a} 

\question \textbf{Which measure is suitable for open-ended distribution?}
\choice{Median}{Mode}{Geometric Mean}{Arithmetic mean}{b}

\question \textbf{Which is not a measure of central tendency?}
\choice{Arithmetic mean}{Mode}{Range}{Quadratic mean}{c}

\question \textbf{Which time series component represents fluctuations occurring at regular intervals within a year?}  
\choice{Trend}{Seasonal Variation}{Irregular Variation}{Cyclic Variation}{b}

\textbf{Answer the next three questions based on the following table:}

\textbf{The following table shows the monthly sales revenue (in thousand dollars) of a store over seven months.}

\begin{table}[H]
\centering
\begin{tabular}{c|c|c|c|c|c|c|c}
Month  & Jan & Feb & Mar & Apr & May & Jun & Jul \\ \hline
Revenue (000\$) & 50  & 55  & 60  & 70  & 75  & 80  & 85  
\end{tabular}
\end{table}

\question \textbf{Which month had the highest sales revenue?}
\choice{May}{Jun}{Jul}{Apr}{c}

\question \textbf{What is the first value of the 2-monthly moving average?}
\choice{52.5}{55}{57.5}{60}{a}

\question \textbf{Using the semi-average method, what is the first average revenue?}
\choice{57.5}{60}{62.5}{65}{b}

\question \textbf{Which business is most likely to experience strong seasonal variation in its sales?}  
\choice{A supermarket}{A toy store}{A furniture store}{A gas station}{b} 


\textbf{Answer the next three questions based on the following information}

\textbf{The following table shows weekly production of milk (in liters) by different varieties of cows.}

\begin{center}
\begin{table}[h]
\centering
\begin{tabular}{c|c|c|c|c|c|c}
Interval     & 10-20 & 20-30 & 30-40 & 40-50 & 50-60 & 60-70 \\ \hline
Frequency    & 5     & 12    & 18    & 25    & 20    & 10    
\end{tabular}
\end{table} 
\end{center}

\question \textbf{What is the median?}
\choice{43}{44}{45}{50}{b}

\question \textbf{What is the lower limit of class interval for first quartile?}
\choice{10}{20}{30}{40}{c}

\question \textbf{What is the 3rd quartile?}
\choice{55.75}{43.75}{53.15}{53.75}{d}

\question 
\textbf{For two non-zero positive numbers, the harmonic mean is 10 and the arithmetic mean is 25. What is the geometric mean?}

\choice{15}{20}{25}{30}{a}

\question \textbf{In Bangladesh, which ministry present the budget?}
\choice{Planning}{Education}{Finance}{Agriculture}{c}

\question \textbf{Which one represents an infinite population?}
\choice{Books in a library}{Fish in the Pacific Ocean}
{Members of a sports club}{Mobile phones in a city}{b}

\question \textbf{Which statistical method requires bivariate or multivariate data?}  
\choice{Standard deviation}{Histogram}{Regression analysis}{Median}{c} 

\question \textbf{Population census is --}
\choice{Official statistics}{Non-official statistics}
{Semi-official statistics}{None of the above}{c}

\question \textbf{Mode is --}  

i. The most frequently occurring value \\  
ii. Unaffected by extreme values \\  
iii. Always unique in a dataset  

\textbf{Which one is correct?}  

\choice{i and ii}{i and iii}{ii and iii}{i, ii and iii}{a}  

\question \textbf{The arithmetic mean of a variable is 10. What is the 
first raw moment around 0?}
\choice{10}{-2}{0}{8}{a}

\question \textbf{The standard deviation of a mesokurtik distribution is 2. What is the value of the 4th central moment?}
\choice{4}{8}{16}{48}{d}

\question \textbf{If $f_i = 3, 5, 7$ and $x_i = 2, 4, 6$, find $\displaystyle \sum_{i=1}^3 f_ix_i^2$.}  
\choice{260}{280}{344}{320}{c}  


\end{questions}

 \vspace{2.5cm}

\begin{center}
“Information is the oil of the 21st century, and analytics is the combustion engine.” - Peter Sondergaard
\end{center}

\pagebreak
%\newpage  %Uncomment to put on new age
\bigskip

\begin{multicols}{3}
[
Answer Key
]
\showallanswers % Phil Hirschorn
\end{multicols}


\end{document}