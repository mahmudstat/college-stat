\documentclass{exam}
%\documentclass[11pt,a4paper]{exam}
\usepackage{amsmath,amsthm,amsfonts,amssymb,dsfont}
\usepackage{ifthen}
\usepackage{setspace}
\usepackage{geometry}
\geometry{
legalpaper, total={177.8mm, 290mm},left=20mm,
top=10mm, bottom=20mm,
}
\usepackage{enumerate}% http://ctan.org/pkg/enumerate
\usepackage{multicol}
\usepackage{hhline}
\usepackage[table]{xcolor}


% Accumulate the answers. Unmodified from Phil Hirschorn's answer
% https://tex.stackexchange.com/questions/15350/showing-solutions-of-the-questions-separately/15353
\newbox\allanswers
\setbox\allanswers=\vbox{}

\newenvironment{answer}
{%
    \global\setbox\allanswers=\vbox\bgroup
    \unvbox\allanswers
}%
{%
    \bigbreak
    \egroup
}

\newcommand{\showallanswers}{\par\unvbox\allanswers}
% End Phil's answer


% Is there a better way?
\newcommand*{\getanswer}[5]{%
    \ifthenelse{\equal{#5}{a}}
    {\begin{answer}\thequestion. (a)~#1\end{answer}}
    {\ifthenelse{\equal{#5}{b}}
        {\begin{answer}\thequestion. (b)~#2\end{answer}}
        {\ifthenelse{\equal{#5}{c}}
            {\begin{answer}\thequestion. (c)~#3\end{answer}}
            {\ifthenelse{\equal{#5}{d}}
                {\begin{answer}\thequestion. (d)~#4\end{answer}}
                {\begin{answer}\textbf{\thequestion. (#5)~Invalid answer choice.}\end{answer}}}}}
}

\setlength\parindent{0pt}
%usage \choice{ }{ }{ }{ }
%(A)(B)(C)(D)
\newcommand{\fourch}[5]{
    \par
    \begin{tabular}{*{4}{@{}p{0.23\textwidth}}}
        (a)~#1 & (b)~#2 & (c)~#3 & (d)~#4
    \end{tabular}
    \getanswer{#1}{#2}{#3}{#4}{#5}
}

%(A)(B)
%(C)(D)
\newcommand{\twoch}[5]{
    \par
    \begin{tabular}{*{2}{@{}p{0.46\textwidth}}}
        (a)~#1 & (b)~#2
    \end{tabular}
    \par
    \begin{tabular}{*{2}{@{}p{0.46\textwidth}}}
        (c)~#3 & (d)~#4
    \end{tabular}
    \getanswer{#1}{#2}{#3}{#4}{#5}
}

%(A)
%(B)
%(C)
%(D)
\newcommand{\onech}[5]{
    \par
    (a)~#1 \par (b)~#2 \par (c)~#3 \par (d)~#4
    \getanswer{#1}{#2}{#3}{#4}{#5}
}

\newlength\widthcha
\newlength\widthchb
\newlength\widthchc
\newlength\widthchd
\newlength\widthch
\newlength\tabmaxwidth

\setlength\tabmaxwidth{0.96\textwidth}
\newlength\fourthtabwidth
\setlength\fourthtabwidth{0.25\textwidth}
\newlength\halftabwidth
\setlength\halftabwidth{0.5\textwidth}

\newcommand{\choice}[5]{%
\settowidth\widthcha{AM.#1}\setlength{\widthch}{\widthcha}%
\settowidth\widthchb{BM.#2}%
\ifdim\widthch<\widthchb\relax\setlength{\widthch}{\widthchb}\fi%
    \settowidth\widthchb{CM.#3}%
\ifdim\widthch<\widthchb\relax\setlength{\widthch}{\widthchb}\fi%
    \settowidth\widthchb{DM.#4}%
\ifdim\widthch<\widthchb\relax\setlength{\widthch}{\widthchb}\fi%

% These if statements were bypassing the \onech option.
% \ifdim\widthch<\fourthtabwidth
%     \fourch{#1}{#2}{#3}{#4}{#5}
% \else\ifdim\widthch<\halftabwidth
% \ifdim\widthch>\fourthtabwidth
%     \twoch{#1}{#2}{#3}{#4}{#5}
% \else
%      \onech{#1}{#2}{#3}{#4}{#5}
%  \fi\fi\fi}

% Allows for the \onech option.
\ifdim\widthch>\halftabwidth
    \onech{#1}{#2}{#3}{#4}{#5}
\else\ifdim\widthch<\halftabwidth
\ifdim\widthch>\fourthtabwidth
    \twoch{#1}{#2}{#3}{#4}{#5}
\else
    \fourch{#1}{#2}{#3}{#4}{#5}
\fi\fi\fi}


\begin{document}

\begin{table}[h]
\centering
\begin{tabular}{lllll}
\textbf{\large SYLHET CADET COLLEGE} &  &  &  &  \\ \cline{4-5} 
FIRST TERM-END EXAMINATION - 2024 &  & \multicolumn{1}{l|}{} & \multicolumn{1}{l|}{Set} & \multicolumn{1}{l|}{A} \\ \cline{4-5} 
CLASS: XII &  &  &  &  \\ \cline{3-5} 
MULTIPLE CHOICE QUESTIONS & \multicolumn{1}{l|}{\textbf{Subject Code:}} & \multicolumn{1}{l|}{1} & \multicolumn{1}{l|}{3} & \multicolumn{1}{l|}{0} \\ \cline{3-5} 
STATISTICS FIRST PAPER &  &  &  &  \\
TIME – 20 minutes &  &  &  &  \\
FULL MARKS – 25 &  &  &  & 
\end{tabular}
\end{table}
%  \normalfont\normalsize
 % 11.45a.m.~--~1.45p.m.
\hrule

\begin{center}
[N.B. – Answer all the questions. Each question carries ONE mark. Block fully, with a black ball- point pen, the circle of the letter that stands for the correct/best answer in the “Answer sheet” for the Multiple Choice Questions Examination.]\\

  
  \textbf{Candidates are asked not to leave any mark or spot on the question paper.}
\end{center}

\textbf{Short Questions}

\onehalfspacing % Set line spacing to 1.5

\begin{enumerate}

\item $P(\bar A) = 1-$ \noindent\rule{2cm}{0.4pt}. 

\item A die is tossed 5 times. What is the probability that 3 comes up 4 times? \noindent\rule{2cm}{0.4pt}

\item What is the range of probability? \noindent\rule{2cm}{0.4pt}

\item If card is drawn from a deck of cards, what is the probability 
that it is not an ace? \noindent\rule{2cm}{0.4pt}

\item Give an example of a discrete variable. \noindent\rule{5cm}{0.4pt}

\item Can $F(x)$ be less than $P(x)$? \noindent\rule{2cm}{0.4pt} 

\item Integration is used if we have a  \noindent\rule{4cm}{0.4pt} random variable.

\item Expectation is equal to  \noindent\rule{2cm}{0.4pt} .

\item $E(a) = $  \noindent\rule{2cm}{0.4pt}?

\item Express $E(X^2)$ in terms of $E(X)$ and $V(X)$  \noindent\rule{2cm}{0.4pt}

\end{enumerate}

\singlespacing % Reset line spacing to 1 from here on

\textbf{Multiple Choice Questions}

\begin{questions}

\question \textbf{In how many ways can a team of 2 be formed from 4 people?}
\choice{4}{6}{8}{12}{b}

\question \textbf{The probability of two disjoint sets happening together is:}
\choice{0.5}{0}{1}{$0  \leq x < 1$}{b}

\question \textbf{The third axiom of probability is --}
\choice{$0 \le P(A) \le 1$}{$P(S) = 1$}{$\displaystyle P(A_1 U A_2 U \cdots U
A_n) = \sum_{i=1}^{\infty}P(A_i)$}{$P(A) = 1 - P(A)$}{c}

\question \textbf{Possible value of probability}

i. -1 \quad
ii. 0.5 \quad
iii. 0

\textbf{Which one is correct?}

\choice{i and ii}{i and iii}{ii and iii}{i, ii and iii}{c}

\question \textbf{There are 3 red, 4 black, and 5 white balls in an urn. If two balls are randomly taken, what is the probability that both are red?}
\choice{$\frac{1}{66}$}{$\frac{1}{22}$}{$\frac{2}{22}$}{$\frac{3}{11}$}{b}

\textbf{Answer the next two questions based on the following information}

\begin{table}[h]
\centering
\begin{tabular}{cccc}
X & 0 & 1 & 2 \\ \hline
P(x) & $\frac 12$ & $\frac14$ & $\frac14$
\end{tabular}
\end{table}

\question \textbf{What is F(1)}
\choice{$0.65$}{$0.75$}{$0.5$}{$1$}{b}

\question \textbf{$P(X \le 1 \le 3) =$--}
\choice{0.75}{0.70}{0.95}{1}{a}

\question \textbf{How many types of random variables are there?}
\choice{2}{3}{4}{5}{a}

\question \textbf{Which of the following is not a discrete random variable?}
\choice{umber of students}{Weight}{Number of heads in coin toss}{Population}{b}

\textbf{Answer the next two questions based on the following information.}

\begin{table}[h]
\begin{center}
\begin{tabular}{|l|l|l|l|l|l|l|}
\hline
x    & 4         & 5         & 6         & 3         & 2         & 1         \\ \hline
P(X) & $\frac16$ & $\frac16$ & $\frac16$ & $\frac16$ & $\frac16$ & $\frac16$ \\ \hline
\end{tabular}
\end{center}
\end{table}

\question \textbf{The value of $P(3<X<5)$ is:}
\choice{$\frac12$}{$\frac16$}{$\frac13$}{0}{b}

\question \textbf{$P(x \neq 2) is:$}
\choice{$\frac56$}{$0$}{1}{Can't be found from this information}{a}

\question \textbf{What is the value of $V(5)$?}
\choice{0}{25}{5}{1}{a}

\textbf{Answer the next THREE questions based on the following information}

\begin{table}[h]
\centering
\begin{tabular}{cccc}
X & 0 & 1 & 2 \\ \hline
P(x) & $\frac 13$ & $\frac14$ & $\frac5{12}$
\end{tabular}
\end{table}

\question \textbf{What is the value of $E(X)$}
\choice{$\frac{15}{12}$}{$\frac{13}{12}$}{$\frac{1}{12}$}{$\frac{11}{13}$}{b}

\question \textbf{What is the value of $E(X^2)$}
\choice{$\frac{25}{12}$}{$\frac{13}{12}$}{$\frac{23}{12}$}{$\frac{25}{13}$}{b}

\question \textbf{What is $V(2X)$?}
\choice{2.93}{2.91}{1.97}{2.97}{d}


\end{questions}

 \vspace{2.5cm}

\begin{center}
An approximate answer to the right problem is worth a good deal more than an exact answer to an approximate problem. – John Tukey.
\end{center}

%\pagebreak
%\newpage  %Uncomment to put on new age
%\bigskip

%\begin{multicols}{3}
%[
%Answer Key
%]
%\showallanswers % Phil Hirschorn
%\end{multicols}

\newpage
\setcounter{page}{1}

\begin{table}[h]
\centering
\begin{tabular}{lllll}
\textbf{\large SYLHET CADET COLLEGE} &  &  &  &  \\ \cline{4-5} 
FIRST TERM-END EXAMINATION - 2024 &  & \multicolumn{1}{l|}{} & \multicolumn{1}{l|}{Set} & \multicolumn{1}{l|}{D} \\ \cline{4-5} 
CLASS: XII &  &  &  &  \\ \cline{3-5} 
STATISTICS (CREATIVE)& \multicolumn{1}{l|}{\textbf{Subject Code:}} & \multicolumn{1}{l|}{1} & \multicolumn{1}{l|}{3} & \multicolumn{1}{l|}{0} \\ \cline{3-5} 
 SECOND PAPER &  &  &  &  \\
TIME – 2 hours \& 10 minutes &  &  &  &  \\
FULL MARKS – 50 &  &  &  & 
\end{tabular}
\end{table}
%  \normalfont\normalsize
 % 11.45a.m.~--~1.45p.m.

\hrule

\begin{center}
[\textbf{N.B.} – The figures of the right margin indicate full marks. Read the stems carefully and answer the associated questions. Answer any \textbf{FIVE} questions taking at least two from each group.]\\

\end{center}

  \begin{center}
  \textbf{Group--A}
  \end{center}
  
    \begin{enumerate}

  
     \item
	  \textbf{It is observed that 50\% of mails are spam. A software filters spam mail before reaching the inbox. Its accuracy for detecting a spam mail is 99\% and chances of tagging a non-spam mail as spam mail is 5\%.} 
  
  \begin{enumerate}
    \item
	What is a disjoint event? \hfill 1
    \item
	For two independent events, what does the Bayes' theorem reduce to? \hfill 2
    \item  
	What is the probability that a mail is tagged as spam?  \hfill 3
    \item
	If a certain mail is tagged as spam, find the probability that it is not a spam mail. \hfill 4
  \end{enumerate}
  
     \item
	  \textbf{A continuos random variable X follows the following probability density function (pdf).} 
	  \begin{center}
	  $f(x) = 6x(1-x); 0\le x\le 1$
  \end{center}
  
  \begin{enumerate}
    \item
	Give an example of a continous random variable. \hfill 1
    \item
	Examine whether the given function is a pdf. \hfill 2
    \item  
	If $P(X>a) = P(X<a)$, find the value of a. \hfill 3
    \item
	Should $P(0.5 \le X \le 1)$  be equal to 0.5? \hfill 4
  \end{enumerate}
  
     \item
	  \textbf{A box contains four blue and 6 green balls. 3 balls are drawn randomly.} 
  
  \begin{enumerate}
    \item
	What is the value of $^nC_r$? \hfill 1
    \item
	Illustrate the difference between permutation and combination with an example. \hfill 2
    \item  
	What is the probability that all balls are green? \hfill 3
    \item
	What is the probabilith that one ball has a different color? \hfill 4
  \end{enumerate}

  
    \begin{center}
  \textbf{Group--B}
  \end{center}
  
       \item
  \textbf{The joint probability function of two random variables X and Y is given below:}
  
  \begin{center}
  $\displaystyle P(X,Y) = \frac {x+2y}{16}; x = 0, 1; y = 0 ,1,2,3$
 \end{center}
 
  \begin{enumerate}
    \item
	Write down the formula of conditional proibability. \hfill 1
    \item
    	What is the relationship between marginal and joint probability? \hfill 2
    \item
    	Find P(X). \hfill 3
     \item
     	Find $P(X\vert Y)$ and $P(X\vert 0)$. \hfill 4
  \end{enumerate}
  
     \item
	  \textbf{The probability distribution of a random X is provided below:} 
	  
	  \begin{table}[h]
	  \centering
\begin{tabular}{c|ccccc}
X & -1 & 0 & 1 & 2 & 3 \\ \hline
P(x) & $\frac 3{20}$ & $\frac 15$ & $\frac 14$ & $\frac 14$ & $\frac 3{20}$
\end{tabular}
\end{table}
  
  \begin{enumerate}
    \item
	What is the expectation of a constant m? \hfill 1
    \item
	Find $E(X).$ \hfill 2
    \item  
	Find $E(Y)$, where $Y = \frac X2$  \hfill 3
    \item
	Find Variance of (2X+3). \hfill 4
  \end{enumerate}
  
  \item
	  \textbf{An umbrella seller earns a revenue of BDT. 5000 if it rains. If it does not rain, he loses BDT. 1000. The probability that it rains on a given day is 0.04.} 
  
  \begin{enumerate}
    \item
	Write down the formula of Expectation for a continuous random variable. \hfill 1
    \item
	Can the value of Expectation be zero? \hfill 2
    \item  
	What is the umbrella seller's expected revenue? \hfill 3
    \item
	What should be the minimum probability of raining for him to achieve revenue greater than zero? \hfill 4
  \end{enumerate}
  
  
    \end{enumerate}

\end{document}