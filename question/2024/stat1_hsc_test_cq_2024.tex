\documentclass{article}
\usepackage{geometry}
\usepackage{amsfonts}

\geometry{
legalpaper, total={177.8mm, 290mm},left=20mm,
top=27mm, bottom=27mm,
}

\begin{document}

\begin{table}[h]
\centering
\begin{tabular}{lllll}
\textbf{\large SYLHET CADET COLLEGE} &  &  &  &  \\ \cline{4-5} 
TEST EXAMINATION - 2024 &  & \multicolumn{1}{l|}{} & \multicolumn{1}{l|}{Set} & \multicolumn{1}{l|}{A} \\ \cline{4-5} 
CLASS: XII &  &  &  &  \\ \cline{3-5} 
STATISTICS (CREATIVE)& \multicolumn{1}{l|}{\textbf{Subject Code:}} & \multicolumn{1}{l|}{1} & \multicolumn{1}{l|}{2} & \multicolumn{1}{l|}{9} \\ \cline{3-5} 
 FIRST PAPER &  &  &  &  \\
TIME – 2 hours \& 35 minutes &  &  &  &  \\
FULL MARKS – 50 &  &  &  & 
\end{tabular}
\end{table}
%  \normalfont\normalsize
 % 11.45a.m.~--~1.45p.m.

\hrule

\begin{center}
[\textbf{N.B.} – The figures of the right margin indicate full marks. Read the stems carefully and answer the associated questions. Answer any \textbf{FIVE} questions taking at least two from each group.]\\

\end{center}

  \begin{center}
  \textbf{Group--A}
  \end{center}
    \begin{enumerate}
   \item
	  \textbf{The capital and profit (in million BDT} of some Bangladeshi industries are bgiven below: 
	  
	  \begin{table}[h]
	  \centering
\begin{tabular}{c|ccccc}
Industry & 1 & 2 & 3 & 4 & 5 \\ \hline
Capital (X) & 20 & 15 & 26 & 31 & 18 \\ \hline
Profit (Y) & 15 & 10 & 17 & 25 & 10
\end{tabular}
\end{table}
  
  \begin{enumerate}
    \item
	What is finite population? \hfill 1
    \item
	What are the functions of statistics? \hfill 2
    \item  
	Find the value of $\displaystyle \sum_{i=1}^5 \sum_{j=1}^5 (x_i - y_j)$ \hfill 3
    \item
	Analyze the statement theoretically and empirically:  $\displaystyle \sum_{i=1}^5 (4x_i-6y_j) = 4 \sum_{i=1}^5 x_i - 6 \sum_{i=1}^5 y_j $ \hfill 4
  \end{enumerate}
  
  
 \item
	  \textbf{An arithmetic series is formed as follows:}
	  
	   \begin{center}
  \textbf{$a, a+c, a+2c, \cdots, a+2nc$}
  \end{center}
  
  \begin{enumerate}
    \item
	What is change of origin and scale? \hfill 1
    \item
	Convert the series into a set of natural numbers. \hfill 2
    \item  
	Find the arithmetic mean of the series with the use of change of origin and scale. \hfill 3
    \item
	Find the geometric mean of the series: $1, 2, 4, \cdots , 2^n$ . \hfill 4
  \end{enumerate}
  
\item
	  \textbf{Arithmetic (AM) and Harmonic Mean (HM) of two numbers are 25 and 9, respectively.} 
    \begin{enumerate}
    \item
	When is HM useful? \hfill 1
    \item
	Derive HM formula using the concept of average velocity. \hfill 2
    \item  
	Find the two values from the stem. \hfill 3
    \item
	Show mathematically that $HM \le AM$ (for n=2) \hfill 4
  \end{enumerate}
 
   \item
	  \textbf{A researcher has collected some data:}
	  \begin{center}
	  $x_1=15, x_2=-12, x_3=10, x_4=21, x_5=33$
  \end{center}
  \begin{enumerate}
    \item
	What is sample? \hfill 1
    \item
	Briefly explain shift or origin and scale. \hfill 2
    \item  
	Compute the value of $\displaystyle \sum_{i=1}^5 (x_i-20)^2$ \hfill 3
    \item
	Find the value of $\displaystyle \sum_{i=1}^5 (3x_i^2-2x_i-3)$ and examine its dependency on origin and scale. \hfill 4
  \end{enumerate} 
  
    \begin{center}
  \textbf{Group--B}
  \end{center}
  
  \item
	  \textbf{The first four moments of a distribution around 5 are 2, 20, 40, and 50, respectively.} 
  
  \begin{enumerate}
    \item
	\item Draw the shape of a left-skewed distribution. \hfill 1
    \item
	\item Derive the value of thew first central moment. \hfill 2 \hfill 2
    \item  
	Obtain the first four central moments. \hfill 3
    \item
	Estimate and comment on the skewness and kurtosis. \hfill 4
  \end{enumerate}
 
 
\item
	  \textbf{United Nations Children's Fund (UNICEF) is an agency of the United Nations responsible for providing humanitarian and developmental aid to children worldwide. A  UNICEF researcher collected heights (in feet) of 7 children for a project, and the heights are} 

	\begin{center}
	  2.2, 2.15, 1.9, 3.1, 2.7, 3.0, 3.5
	  	\end{center}
  
  \begin{enumerate}
    \item
	Which value are central moments estimated around? \hfill 1
    \item
	Moments around origin (0) are central moments - Comment. \hfill 2
    \item  
	Find the first three central moments. \hfill 3
    \item
	Find the skewness of the data and interpret.  \hfill 4
  \end{enumerate}
 
   \item
	  \textbf{Bangladesh foreign debt has been increasing rapidly in recent years. The Bangladesh bank provides the follwoing data.}
	  
	  \begin{table}[h]
	  \centering
\begin{tabular}{c|c|c|c|c|c|c|c|c|c}
Fiscal Year & 2015-16 & 2016-17 & 2017-18 & 2018-19 & 2019-20 & 2020-21 & 2021-22 & 2022-23 & 2023-24 \\ \hline
Debt & 41.17 & 45.81 & 56.01 & 62.63 & 68.55 & 81.62 & 95.45 & 98.94 & $\sim$130.00
\end{tabular}
\end{table}
  
  \begin{enumerate}
    \item
	Name the components of time series. \hfill 1
    \item
	What are linear and non-linear trends? \hfill 2
    \item  
	Find 3-yearly moving avergae from the data and plot. \hfill 3
    \item
	Whaich components of time series may underlie the data? Analyze. \hfill 4
  \end{enumerate}
 
 
   \item
	  \textbf{Globalization creates many opportuninities. Many government and non-government organizatios collcet and analyze data analyze different aspects of the trend.} 
  
  \begin{enumerate}
    \item
	What is official statistics?  \hfill 1
    \item
	What is the role of the United Nations? \hfill 2
    \item  
	What are the limitations of published statistics in Bangladesh? \hfill 3
    \item
	Compare the statistical analysis in Bangladesh with international standard. \hfill 4
  \end{enumerate}
 
  

\end{enumerate}
\end{document}