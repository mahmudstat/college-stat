\documentclass{article}
\usepackage{geometry}
\usepackage{amsfonts}
\usepackage{float}

\geometry{
legalpaper, total={177.8mm, 290mm},left=20mm,
top=27mm, bottom=27mm,
}

\begin{document}

\begin{table}[h]
\centering
\begin{tabular}{lllll}
\textbf{\large SYLHET CADET COLLEGE} &  &  &  &  \\ \cline{4-5} 
FIRST TERM-END EXAMINATION - 2024 &  & \multicolumn{1}{l|}{} & \multicolumn{1}{l|}{Set} & \multicolumn{1}{l|}{A} \\ \cline{4-5} 
CLASS: XII &  &  &  &  \\ \cline{3-5} 
STATISTICS (CREATIVE)& \multicolumn{1}{l|}{\textbf{Subject Code:}} & \multicolumn{1}{l|}{1} & \multicolumn{1}{l|}{3} & \multicolumn{1}{l|}{0} \\ \cline{3-5} 
 FIRST PAPER &  &  &  &  \\
TIME – 2 hours \& 35 minutes &  &  &  &  \\
FULL MARKS – 50 &  &  &  & 
\end{tabular}
\end{table}
%  \normalfont\normalsize
 % 11.45a.m.~--~1.45p.m.

\hrule

\begin{center}
[\textbf{N.B.} – The figures of the right margin indicate full marks. Read the stems carefully and answer the associated questions. Answer any \textbf{FIVE} questions taking at least two questions from each group]\\


\end{center}

  \begin{center}
  \textbf{Group--A}
  \end{center}

  \begin{enumerate}
  
    \item
	  \textbf{As part of an experiment, a neutral coin is tossed 5 times.} 
  
  \begin{enumerate}
    \item
	What is a neutral coin? \hfill 1
    \item
	If a coin is flung n times, show the no. of outcomes generated. \hfill 2
    \item  
	What is the probability of getting a) at least 3 heads, b) at most 3 heads? \hfill 3
    \item
	Are these probabilities equal? a) Getting at least 2 heads \& b) Getting at least 2 tails. \\ Also justify logically. \hfill 4
  \end{enumerate}
   \item
	  \textbf{A sorcerer draws 3 cards from a pack (i) with replacement and then (ii) without replacement. The cards were well-shuffled before drawing.} 
  
  \begin{enumerate}
    \item
	What is an uncertain event? \hfill 1
    \item
	Differentiate between classical and empirical approach of probability.  \hfill 2
    \item  
	As per (i), what is the probability that the cards have different color? \hfill 3
    \item
	As per (ii), what is the probability that the cards are aces of same color?  \hfill 4
  \end{enumerate}
  
     \item
	  \textbf{A continuos random variable X follows the following probability density function (pdf).} 
	  \begin{center}
	  $f(x) = 6x(x-1); 0\le x\le 1$
  \end{center}
  
  \begin{enumerate}
    \item
	Give an example of a continous random variable. \hfill 1
    \item
	Examine whether the given function is a pdf. \hfill 2
    \item  
	If $P(X>a) = P(X<a)$, find the value of a. \hfill 3
    \item
	Should $P(0.5 \le X \le 1)$  be equal to 0.5? \hfill 4
  \end{enumerate}

        \item \textbf{A random variable is distributed as below:}
        
        \begin{center}
  \textbf{$P(X) = \frac{3-\vert 4-x\vert}{k}; x=2,3,4,5,6$}
  \end{center}

  \begin{enumerate}
    \item
	What is the Expectation equivalent to? \hfill 1
    \item
    	Find the value of k. \hfill 2
    \item
    	Determine the value of the expectation. \hfill 3
     \item
     	Find $V(2X-1)$ \hfill 4
  \end{enumerate}
  
    \begin{center}
  \textbf{Group--B}
  \end{center}
  
   \item
	  \textbf{A random variable is distributed as follows:} 
	  
	  \begin{table}[h]
	  \centering
\begin{tabular}{ccccccc}
Value & 0 & 1 & 2 & 3 & 4 & 5 \\ \hline
Frequency & 70 & 73 & 27 & 15 & 4 & 1
\end{tabular}
\end{table}

  \begin{enumerate}
    \item
	What is the mean of Poisson distribution? \hfill 1
    \item
	What is the relationship between mean and standard deviation of Poisson distribution? \hfill 2
    \item  
	Find the mean and variance of the given distribution. \hfill 3
    \item
	Compare the observed and expected frequencies, assuming a Possion distribution. \hfill 4
  \end{enumerate}
  
   \item
	  \textbf{The number of defective pen produced by a company follows a binomial distribution with expectation 1.5 and variance 1.125.}. 
  
  \begin{enumerate}
    \item
	What is the mean of binomial distribution \hfill 1
    \item
	Can variance be greater than mean in binomial distribution? \hfill 2
    \item  
	Determine the probability function of the number of defective items produced by the company. \hfill 3
    \item
	What is the probability that the number of defective items is no less than 3? \hfill 4
  \end{enumerate}
  
    \item
	  \textbf{The number of customers coming at a shop per minute follows a Poisson distribution,  whose mean is 3.} 
  
  \begin{enumerate}
    \item
	What is a Poisson variate? \hfill 1
    \item
	Can the mean of Poisson distribution be negative? \hfill 2
    \item  
	Find the probability that the number of customers coming is between 1 and 2. \hfill 3
    \item
	Analyze the statement: P(X=2) = P(X=3).  \hfill 4
  \end{enumerate}
  

     \item
	  \textbf{As part of an analysis, a researcher collected data on women and live births.} 
	  \begin{table}[H]
	  \centering
\begin{tabular}{c|c|c|c|c|c|c|c}
Age & 15-19 & 20-24 & 25-29 & 30-34 & 35-39 & 40-44 & 45-49 \\ \hline
No. of Women & 540 & 760 & 530 & 495 & 450 & 505 & 430 \\ \hline
No. of live births & 109 & 198 & 86 & 90 & 65 & 76 & 60
\end{tabular}
\end{table}
  
  \begin{enumerate}
    \item
	What is the formula of death rate? \hfill 1
    \item
	Write down the uses of vital statistics. \hfill 2
    \item  
	Find teh Age Specific Birth Rates (ASFR). \hfill 3
    \item
	Find the GFR and compare its concept and value with ASFRs. \hfill 4
  \end{enumerate}
  
\end{enumerate}
\end{document}