\documentclass{article}
\usepackage{geometry}
\usepackage{amsfonts}

\geometry{
a4paper, total={170mm, 257mm},left=20mm,
top=20mm,
}

\begin{document}

\begin{center}
  \bfseries\large
  Sylhet Cadet College

\normalsize
  Model Test Examination - 2022

  Class: HSC

  Subject: Statistics First Paper (Creative)

  Time: 1 hour \& 40 minutes \qquad \qquad \qquad Subject Code: 129  \qquad  \qquad \qquad Full Marks: 30

%  \normalfont\normalsize
 % 11.45a.m.~--~1.45p.m.
\end{center}

\noindent
\begin{tabular}{p{\dimexpr\linewidth-2\tabcolsep}}
  Answer three questions taking at least 1 (one) from each group. Figures in the right indicate full marks.\\
  \hline
\end{tabular}

\begin{center}
\textbf{Group A}
\end{center}

\begin{enumerate}

   \item
	  \textbf{Below are some information} 
	  
 \begin{center}
  $x_1=3, x_2=4, x_3=1, x_4=0$ \\
	  $y_1=1, y_2=5, y_3=0, y_4=2$
  \end{center}
  
  \begin{enumerate}
    \item
	What is a qualitative variable? \hfill 1
    \item
	Find $\displaystyle \sum_{i=1}^{4}x_i^2$ \hfill 2
    \item  
	Prove that $\displaystyle \sum_{i=1}^{4} (x_i+y_i) = \sum_{i=1}^{4}x_i + \sum_{i=1}^{4}y_i $ \hfill 3
    \item
	Find the value of $\displaystyle \sum_{i=1}^{4} x_iy_i-\sum_{i=1}^{4} x_i+4$ \hfill 4

  \end{enumerate}

     \item
	  \textbf{For a particular data set, Median = 120, Mode = 110, Standard Deviation = 4, and Coefficient of Variation (CV)  = 3.2} 
  
  \begin{enumerate}
    \item
	Why is  CV used?  \hfill 1
    \item
	Find arithmetic mean. \hfill 2
    \item  
	Find skewness according to Pearson's method ($SK_P$) \hfill 3
    \item
	Does $SK_P$ convey the proper idea about the data as to the given information? Justify. \hfill 4
  \end{enumerate}
  
  
   \item
	  \textbf{A clyclist moves around a square-shaped lake with the speeds 20, 25, 30, and 16 km per hour.} 
  
  \begin{enumerate}
    \item
	What is grouped data? \hfill 1
    \item
	Is arithmetic mean suitable for this data? \hfill 2
    \item  
	Find the average speed of the cyclist. \hfill 3
    \item
	Can we use some other formula for finding the average? Demonstrate. \hfill 4
  \end{enumerate}
  
   \item
	  \textbf{Grades of a an undergraduate student with major in statistics are given} 

\begin{table}[h]
\centering
\begin{tabular}{c|c|c}
\hline
Course & Grade & Credit \\ \hline
Probability & 3.75 & 4 \\ 
Simulation & 3.50 & 3 \\ 
Calculas & 3.50 & 4 \\ 
Linear Algebra & 3.75 & 4 \\ 
Econometrics & 3.00 & 2 \\ 
Programming & 3.50 & 3 \\ \hline
\end{tabular}
\end{table}

  
  \begin{enumerate}
    \item
	Write down the formula of weighted mean. \hfill 1
    \item
	What is difference between weight and frequency? \hfill 2
    \item  
	Determine the GPA of the student. \hfill 3
    \item
	Determine the geometric mean for the data and evaluate \\ suitability. \hfill 4
  \end{enumerate}

\begin{center}
\textbf{Group B}
\end{center}

   \item
	  \textbf{Marks obtained by a student in 7 subjects are} 
	  \begin{center}
	  70, 66, 55, 45, 80, 30, 82
	\end{center}
  
  \begin{enumerate}
    \item
	What is negative skewness? \hfill 1
    \item
	Draw graphs of positive and negative skewness showing the locations of mean and median. \hfill 2
    \item  
	Determine the five number summary from the stem and explain. \hfill 3
    \item
	Are the data symmetric? If not, comment on the pattern of data. \hfill 4
\end{enumerate}
  
     \item
	  \textbf{Goals scored by Karim Benzema in five seasons are recorded to be the following:} 
	  
	  \begin{table}[h]
	  \centering
\begin{tabular}{c|c|c}
Season & La Liga (x) & Uefa Champions League (y) \\ \hline
2017-18 & 5 & 5 \\ 
2018-19 & 21 & 4 \\
2019-20 & 21 & 5 \\
2020-21 & 23 & 6 \\ 
2021-22 & 27 & 15 \\ \hline
\end{tabular}
\end{table}
  
  \begin{enumerate}
    \item
	What is a quantitative variable? \hfill 1
    \item
	What is the notation to denote his total number of goals? \hfill 2
    \item  
	Compute $\displaystyle \sum_{i=1}^5 (y_i - 3)^2$ \hfill 
    \item
	Find total number of goals using two different notations and examine whether they match. \hfill 4
  \end{enumerate}
  
     \item
	  \textbf{Given below is a series of data.} 
	  
	  	    \begin{center}
	 $5, 7, 9, \cdots , 123$
	    \end{center}
  
  \begin{enumerate}
    \item
	What is the summation of natural numbers up to nth value? \hfill 1
    \item
	Find the arithmetic mean of natural numbers from 1 up to 20. \hfill 2
    \item  
	Find the arithmetic mean of the given series. \hfill 3
    \item
	Prove that arithmetic mean is greater than gemetric mean theoretically and empricially. \hfill 4
  \end{enumerate}
  
   \item
	  \textbf{In ODI cricket, two top batsmen are (as of 2nd Sept, 2022) Babar Azam and Rassie van der Dussen. Their average (arithmetic mean) scores are 59.79 and 69.32, appearing in 90 (including being not out in 12 occassions) and 33 (including being not out in 11 occassions) matches, respectively.} 
  
  \begin{enumerate}
    \item
	When is arithmetic mean inappropriate to use? \hfill 1
    \item
	Is arithmetic mean always suitable for comparison? \hfill 2
    \item  
	Find the combined arithmetic mean and explain. \hfill 3
    \item
	How to compare two sets of data having significantly distinct ranges? \hfill 4
  \end{enumerate}

\end{enumerate}

\end{document}