\documentclass{exam}
%\documentclass[11pt,a4paper]{exam}
\usepackage{amsmath,amsthm,amsfonts,amssymb,dsfont}
\usepackage{ifthen}
\usepackage{enumerate}% http://ctan.org/pkg/enumerate
\usepackage{multicol}



% Accumulate the answers. Unmodified from Phil Hirschorn's answer
% https://tex.stackexchange.com/questions/15350/showing-solutions-of-the-questions-separately/15353
\newbox\allanswers
\setbox\allanswers=\vbox{}

\newenvironment{answer}
{%
    \global\setbox\allanswers=\vbox\bgroup
    \unvbox\allanswers
}%
{%
    \bigbreak
    \egroup
}

\newcommand{\showallanswers}{\par\unvbox\allanswers}
% End Phil's answer


% Is there a better way?
\newcommand*{\getanswer}[5]{%
    \ifthenelse{\equal{#5}{a}}
    {\begin{answer}\thequestion. (a)~#1\end{answer}}
    {\ifthenelse{\equal{#5}{b}}
        {\begin{answer}\thequestion. (b)~#2\end{answer}}
        {\ifthenelse{\equal{#5}{c}}
            {\begin{answer}\thequestion. (c)~#3\end{answer}}
            {\ifthenelse{\equal{#5}{d}}
                {\begin{answer}\thequestion. (d)~#4\end{answer}}
                {\begin{answer}\textbf{\thequestion. (#5)~Invalid answer choice.}\end{answer}}}}}
}

\setlength\parindent{0pt}
%usage \choice{ }{ }{ }{ }
%(A)(B)(C)(D)
\newcommand{\fourch}[5]{
    \par
    \begin{tabular}{*{4}{@{}p{0.23\textwidth}}}
        (a)~#1 & (b)~#2 & (c)~#3 & (d)~#4
    \end{tabular}
    \getanswer{#1}{#2}{#3}{#4}{#5}
}

%(A)(B)
%(C)(D)
\newcommand{\twoch}[5]{
    \par
    \begin{tabular}{*{2}{@{}p{0.46\textwidth}}}
        (a)~#1 & (b)~#2
    \end{tabular}
    \par
    \begin{tabular}{*{2}{@{}p{0.46\textwidth}}}
        (c)~#3 & (d)~#4
    \end{tabular}
    \getanswer{#1}{#2}{#3}{#4}{#5}
}

%(A)
%(B)
%(C)
%(D)
\newcommand{\onech}[5]{
    \par
    (a)~#1 \par (b)~#2 \par (c)~#3 \par (d)~#4
    \getanswer{#1}{#2}{#3}{#4}{#5}
}

\newlength\widthcha
\newlength\widthchb
\newlength\widthchc
\newlength\widthchd
\newlength\widthch
\newlength\tabmaxwidth

\setlength\tabmaxwidth{0.96\textwidth}
\newlength\fourthtabwidth
\setlength\fourthtabwidth{0.25\textwidth}
\newlength\halftabwidth
\setlength\halftabwidth{0.5\textwidth}

\newcommand{\choice}[5]{%
\settowidth\widthcha{AM.#1}\setlength{\widthch}{\widthcha}%
\settowidth\widthchb{BM.#2}%
\ifdim\widthch<\widthchb\relax\setlength{\widthch}{\widthchb}\fi%
    \settowidth\widthchb{CM.#3}%
\ifdim\widthch<\widthchb\relax\setlength{\widthch}{\widthchb}\fi%
    \settowidth\widthchb{DM.#4}%
\ifdim\widthch<\widthchb\relax\setlength{\widthch}{\widthchb}\fi%

% These if statements were bypassing the \onech option.
% \ifdim\widthch<\fourthtabwidth
%     \fourch{#1}{#2}{#3}{#4}{#5}
% \else\ifdim\widthch<\halftabwidth
% \ifdim\widthch>\fourthtabwidth
%     \twoch{#1}{#2}{#3}{#4}{#5}
% \else
%      \onech{#1}{#2}{#3}{#4}{#5}
%  \fi\fi\fi}

% Allows for the \onech option.
\ifdim\widthch>\halftabwidth
    \onech{#1}{#2}{#3}{#4}{#5}
\else\ifdim\widthch<\halftabwidth
\ifdim\widthch>\fourthtabwidth
    \twoch{#1}{#2}{#3}{#4}{#5}
\else
    \fourch{#1}{#2}{#3}{#4}{#5}
\fi\fi\fi}


\begin{document}

\begin{center}
  \bfseries\large
  Sylhet Cadet College

\normalsize
  Pre-Test Examination - 2022

  Class: XII
  
    Set - D

  Subject: Statistics First Paper (MCQ)

  Time: 25 minutes \qquad \qquad \qquad \qquad Subject Code: 130   \qquad \qquad \qquad  \qquad Full Marks: 25

%  \normalfont\normalsize
 % 11.45a.m.~--~1.45p.m.
\end{center}

\textbf{Answer all the questions. Each question is worth one (1) mark.}  

\begin{questions}

\question \textbf{If a neutral die is thrown, the probability of having a digit greater than 6 is}
\choice{$\frac 1 6$}{$\frac 0 6$}{$\frac 2 3$}{$\frac 3 6$}{b}

\question \textbf{Tossing a coin twice generates how many outcomes?}
\choice{4}{16}{8}{2}{a}

\question \textbf{The probability of two disjoint sets happening together is:}
\choice{0.5}{0}{1}{$0  \leq x < 1$}{b}

\textbf{Answer the next three question using the following information}

$P(A) = \frac 1 3, P(B) = \frac 1 2 \space \& \space P(A\cup B) = \frac 1 4$

\question \textbf{$P(A \cap B) = ?$}
\choice{$\frac {5}{12}$}{$\frac12$}{$\frac{7}{12}$}{$\frac{15}{16}$}{c}

\question \textbf{$P(A \cap \bar B)=?$}
\choice{$\frac{3}{4}$}{$\frac{5}{6}$}{$\frac{1}{4}$}{$\frac{1}{12}$}{d}

\question \textbf{What is the probability that B occurs or A does not occur?}
\choice{$\frac{3}{4}$}{$\frac{7}{12}$}{$\frac{5}{12}$}{$\frac{1}{3}$}{a}

\question \textbf{An un contains 10 red and 5 black balls. Two balls are drawn; what is the probability of getting two red balls?}
\choice{$\frac 37$}{$\frac 47$}{$\frac {20}{21}$}{$\frac 2{21}$}{a}

\question \textbf{How many types of random variables are there?}
\choice{2}{3}{4}{5}{a}


\question \textbf{If $f(x) = 2x; 0 <x<3, F(3) = ?$}
\choice{3}{0}{1}{2.5}{c}

\question \textbf{Which one is a discrete random variable?}
\choice{Height}{Weight}{Diameter}{Released version number of a software}{d}

\question \textbf{Which one is a property of joint probability distribution?}
\choice{$P(X_i,Y_j)<1$}{$P(X_i,Y_j)=0$}{$P(X_i,Y_j)<0$}{$0 \leq P(X_i,Y_j)\leq 1$}{d}

\textbf{Answer the next three questions based on the following information.}

A card is drawn from a pack of playing cards.

\question \textbf{What is the probability that the card is a King?}
\choice{0.0192}{0.25}{0.5}{0.0769}{d}

\question \textbf{P(The card is not from Diamonds)--}
\choice{$\frac12$}{$0$}{$\frac34$}{$\frac14$}{c}

\question \textbf{P(The card is red or Clubs)}
\choice{$\frac14$}{$\frac12$}{$\frac23$}{$\frac34$}{d}

\question \textbf{If $f(x) = kx^3; -1 \leq x \leq 1$, then k is}

i) positive \\
ii) negative  \\
iii) lies from -1 to 1

\choice{i}{ii}{iii}{i and ii}{a}

\question \textbf{The minimum value of probability is}
\choice{$-\alpha$}{1}{0}{-1}{c}

\question \textbf{Each element of sample space is called--}
\choice{Trial}{Experiment}{Variable}{Sample Point}{d}

\question \textbf{Two events not ocurring together are called--}
\choice{dependent Events}{Independent Events}{Mutually Exclusive Events}{Marginal Events}{c}

\question \textbf{If A and B are independent, which formula is correct?}
\choice{$P(A \cap B) = P(A) \cdot P(B)$}{$P(A \cap B) = P(\bar A) \cdot P(B)$}{$P(A \cap B) = P(A) \cdot P(\bar B)$}{$P(A \cap \bar B) = P(A) \cdot P(B)$}{a}

\textbf{Answer the next two questions based on the following information.}

\begin{table}[h]
\begin{center}
\begin{tabular}{|l|l|l|l|l|l|l|}
\hline
x    & 4         & 5         & 6         & 3         & 2         & 1         \\ \hline
P(X) & $\frac16$ & $\frac16$ & $\frac16$ & $\frac16$ & $\frac16$ & $\frac16$ \\ \hline
\end{tabular}
\end{center}
\end{table}

\question \textbf{The value of $P(3<X<5)$ is:}
\choice{$\frac12$}{$\frac16$}{$\frac13$}{0}{b}

\question \textbf{$P(x \neq 2) is:$}
\choice{$\frac56$}{$0$}{1}{Can't be found from this information}{a}

\question \textbf{Which of the following is not a discrete random variable?}
\choice{Number of students}{Weight}{Number of heads in coin toss}{Population}{b}

\question \textbf{Which one is a property of a probability distribution?}
\choice{$P(x_i) = 0$}{$P(x_i \ne 1)$}{$\Sigma P(x_i) = 1$}{$\int_x P(X) dx \le 1$}{c}

\textbf{Answer the next two questions based on the following information:}

$P(x,y) = \frac 1{21}(x+y); x = 1,2,3$ and $y=1,2$

\question \textbf{P(x)=?}
\choice{$P(x) = \frac{2x+3}{21}$}{$P(x) = \frac{x+3}{27}$}{$P(x) = \frac{4x+3}{21}$}{$P(x) = \frac{2x+5}{21}$}{a}

\question \textbf{P(y)=?}
\choice{$\frac{y+2}{7}$}{$\frac{y+3}{7}$}{$\frac{3y+2}{7}$}{$\frac{y+2}{9}$}{c}
%\question \textbf{To complete the song, the last answer should be
%\choice{a}{b}{c}{d}{e} % Invalid answer choice

\end{questions}

%\newpage  %Uncomment to put on new age
\bigskip

\begin{multicols}{3}
[
Answer Key: (Correction required)
]
\showallanswers % Phil Hirschorn
\end{multicols}


\end{document}