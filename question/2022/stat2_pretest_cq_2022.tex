\documentclass{article}
\usepackage{geometry}
\usepackage{amsfonts}

\geometry{
a4paper, total={170mm, 257mm},left=20mm,
top=20mm,
}

\begin{document}

\begin{center}
  \bfseries\large
  Sylhet Cadet College

\normalsize
  Pre-Test Examination - 2022

  Class: XII
  
  Set - B

  Subject: Statistics 2nd Paper (Creative) 

  Time: 2 hour \& 35 minutes \qquad \qquad \qquad Subject Code: 130  \qquad  \qquad \qquad Full Marks: 50

%  \normalfont\normalsize
 % 11.45a.m.~--~1.45p.m.
\end{center}

\noindent
\begin{tabular}{p{\dimexpr\linewidth-2\tabcolsep}}
  Answer FIVE questions taking at least two (2) from each group. Figures in the right indicate full marks.\\
  \hline
\end{tabular}

\begin{center}
\textbf{Group A}
\end{center}

\begin{enumerate}

 \item
	  \textbf{Sadman has an urn with 5 red and 4 white balls. He has randomly drawn two balls from the urn.} 
  
  \begin{enumerate}
    \item
	What is the probability of an uncertain event? \hfill 1
    \item
	Write the third axiom of probability. \hfill 2
    \item  
	What is the probability that both the balls drawn by Sadman are white? \hfill 3
    \item
	Are the probabilities of both balls being same color and different color equal? Analyze. \hfill 4
  \end{enumerate}

 \item
	  \textbf{$P(A) = \frac{3}{10}, P(B) = \frac 25, P(B\cup A) = \frac12$} 
  
  \begin{enumerate}
    \item
	What is an independent event? \hfill 1
    \item
	What is the relationship between independency and mutual excluvity? \hfill 2
    \item  
	Find $P(A \vert B)$ and $P(B \vert A)$ \hfill 3
    \item
	Verify the equality mathematically \& empirically: $P(B) = P(A) \cdot P(B \vert A) + P(\bar A) \cdot P(B \vert \bar A)$ \hfill 4
  \end{enumerate}

 \item
  \textbf{The probability density function of a continuous random variable is}

$f(x) =kx^2+kx+ \frac 18; 0 \le x \le 2 $

  \begin{enumerate}
    \item
	What is a continuous random variable? \hfill 1
    \item
    	Find the value of k \hfill 2
    \item
    	Find the probability that the values of x would lie between 1 and 3. \hfill 3
     \item
     	Find $P(1 \le X \le 3)$  \hfill 4
  \end{enumerate}

 \item
	  \textbf{The joint probability function of two random variables X \& Y is given below:}

$P(x,y) = \frac{1}{21}(x+y); x=1,2,3$ \& $y = 1,2$ 
  
  \begin{enumerate}
    \item
	What is a probability density function (pdf)? \hfill 1
    \item
	What is P(X=a) in a pdf, where a is an aribitrary number? \hfill 2
    \item  
	Find the marginal probabilities. \hfill 3
    \item
	Find $P(x \vert y), P(x \vert 1)$ and $P(y|4)$ \hfill 4
  \end{enumerate}
 
\begin{center}
\textbf{Group B}
\end{center}

 \item
  \textbf{Sakib has recently graduated from the University of Dhaka. he applies to two firms - EduCube \& Digic- for a Data Analyst job. The probability of hiring by EduCube is 0.8 and by Digic is 0.4. The probability that none hires is 0.5.} 
  
  \begin{enumerate}
    \item
	What is a sample space? \hfill 1
    \item
	Explain how to find $P(\bar A \cap B)$ using Venn Diagram. \hfill 2
    \item  
	Find the probability of hirng by by Digic but not by EduCube. \hfill 3
    \item
	Find the probability that no firm will reject him. \hfill 4
  \end{enumerate}

 \item
	  \textbf{Two dice are thrown together. The dice are named A and B.} 
  
  \begin{enumerate}
    \item
	What is P(A=7)? \hfill 1
    \item
	Create the sample space. \hfill 2
    \item  
	What is the probability that the outcomes of A \& B are different? \hfill 3
    \item
	Determine the probability that the summation of outcome of two dice is a prime number. \hfill 4
  \end{enumerate}

 \item
	  \textbf{A magician draws two cards from a pack (i) with replacement and then (ii) without replacement. The cards were well-shuffled before drawing.} 
  
  \begin{enumerate}
    \item
	What is the probability of an impossible event? \hfill 1
    \item
	How to determine the probability of a joint event?  \hfill 2
    \item  
	As per (i), what is the probability that the cards have different color? \hfill 3
    \item
	As per (ii), what is the probability that the cardsare aces of same color?  \hfill 4
  \end{enumerate}

 \item
	  \textbf{The probability distribution of a discrete random variable X is given below:} 

	  \begin{table}[h]
	  \begin{center}
\begin{tabular}{lllllll}
x    & -2   & -1 & 0   & 1 & 3 & 4   \\ \hline
P(x) & 0.1 & k & 2k & 3k & 4k & 0.2
\end{tabular} 
\end{center}	
\end{table}
  
  \begin{enumerate}
    \item
	What is $\Sigma P(x)$? \hfill 1
    \item
	Find the value of k. \hfill 2
    \item  
	Find $P(X \geq 0) \& P(X < 1)$ \hfill 3
    \item
	Find the cumulative distribution function, F(X) and F(2) and explain. \hfill 4
  \end{enumerate}
  \end{enumerate}
\end{document}