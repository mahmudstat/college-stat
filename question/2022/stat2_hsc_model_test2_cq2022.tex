\documentclass{article}
\usepackage{geometry}
\usepackage{amsfonts}

\geometry{
a4paper, total={170mm, 257mm},left=20mm,
top=20mm,
}

\begin{document}

\begin{center}
  \bfseries\large
  Sylhet Cadet College

\normalsize
  Model Test Examination - 2022

  Class: HSC

  Subject: Statistics 2nd Paper (Creative)

  Time: 1 hour \& 40 minutes \qquad \qquad \qquad Subject Code: 129  \qquad  \qquad \qquad Full Marks: 30

%  \normalfont\normalsize
 % 11.45a.m.~--~1.45p.m.
\end{center}

\noindent
\begin{tabular}{p{\dimexpr\linewidth-2\tabcolsep}}
  Answer three questions taking at least 1 (one) from each group. Figures in the right indicate full marks.\\
  \hline
\end{tabular}

\begin{center}
\textbf{Group A}
\end{center}

\begin{enumerate}

 \item
	  \textbf{A red and a blue dice are thrown once. The dice are absolutely neutral and independent.} 
  
  \begin{enumerate}
    \item
	What is a simple event? \hfill 1
    \item
	Give an example of a certain event using set theory. \hfill 2
    \item  
	Find the probability that the difference of two digits from two dices is less than 3.\hfill 3
    \item
	Are the probabilities of getting greater digit from the blue die and that from the red die equal? Justify. \hfill 4
  \end{enumerate}

 \item
	  \textbf{A coin is tossed five times. The number of heads appearing from the tosses is considered a discrete random random variable.} 
  
  \begin{enumerate}
    \item
	What is a discrete random variable? \hfill 1
    \item
	Are probability distributions and frequency distributions similar? Show with an example. \hfill 2
    \item  
	Find the probability distribution from the stem and show in a table.\hfill 3
    \item
	Find the probability that a head will appear in more than 3 tosses.  \hfill 4
  \end{enumerate}

 \item
	  \textbf{A professor showed a probability distribution in a class:}
	  
	  \begin{table}[h]
	  \begin{center}
\begin{tabular}{llllll}
x    & 1   & 2 & 3   & 4 & 5   \\ \hline
p(x) & 0.1 & a & 0.3 & b & 0.2 
\end{tabular} \\
\textbf{The value of the arithmetic mean of the distribution is 3.}
\end{center}	
\end{table}


    
  \begin{enumerate}
    \item
	What is the formula of expectation? \hfill 1
    \item
	What is the variance of a constant? Explain logically. \hfill 2
    \item  
	What are the values of a \& b? \hfill 3
    \item
	Find and explain the variance of the distribution. \hfill 4
  \end{enumerate}
  
       \item
	  \textbf{X is a random variable having the below functional form:} 
  
  $P(X) = \frac{6-|7-x|}{k}; x = 1, 2, \cdots,10$ \\
  Y is another variable having the relationship y = 3x+5
  
  \begin{enumerate}
    \item
	What is joint probability? \hfill 1
    \item
	What is the minimum possible value of variance? Why? \hfill 2
    \item  
	Find the value of k. \hfill 3
    \item
	Find E(X) and E(Y). Why are they different? \hfill 4
  \end{enumerate}
  
  \begin{center}
\textbf{Group B}
\end{center}

   \item
	  \textbf{A survey of Television (TV) users at Gulshan in Dhaka was conducted to find how many sets each family use. The following data were obtained:} 
	  
	  	  \begin{table}[h]
	  \begin{center}
\begin{tabular}{llllll}
No of TV set    & 0 & 1  & 2 & 3    \\ \hline
No of family & 10 & 75 & 10 & 5
\end{tabular}
\end{center}	
\end{table}
  
  \begin{enumerate}
    \item
	What is Expectation equivalent to? \hfill 1
    \item
	Can Variance be negative? Why or why not? \hfill 2
    \item  
	Find the variance of the number of TV sets. \hfill 3
    \item
	Find and compoare between arithmetic mean and expectation. \hfill 4
  \end{enumerate}
  


 \item
	  \textbf{A farmer plans to store rice seeds for future use. It was found that 8 out of 20 seeds are rotten. He then collected a sample of 15 seeds.} 
  
  \begin{enumerate}
    \item
	What is Bernoulli trial? \hfill 1
    \item
	How are Bernoulli and Binomial distributions related? \hfill 2
    \item  
	What is the probability that at least one seed is rotten out of 15? \hfill 3
    \item
	What is the probability that the number of rotten seeds is greater than the arithmetic mean? \hfill 4
  \end{enumerate}
  
   \item
	  \textbf{BTCL receives 2.5 telephone calls on average from 4 pm to 6 pm. The number of calls received is a random variable. } 
  
  \begin{enumerate}
    \item
	When is Poisson variate applicable? \hfill 1
    \item
	Show conversion criteria and method from Binomial to Poisson distribution. \hfill 2
    \item  
	Find the probability of receiving no more than 3 calls. \hfill 3
    \item
	Find the pattern of calls and show on graph paper.  \hfill 4 \\
	Hint: Find probabilities: P(0), P(1), $\cdots$
  \end{enumerate}
  
   \item
	  \textbf{Population of Dhaka and Sylhet by different age groups and areas are given below:} 
	  
	  \begin{table}[h]
	  \begin{center}
\begin{tabular}{lllll}
Division & \multicolumn{3}{c}{Age}         & Area ($km^2$) \\
         & 0-14      & 15-64    & 65+      &               \\
Dhaka    & 10,000,00 & 5,00,000 & 5,80,000 & 1,880         \\
Sylhet   & 7,00,000  & 2,70,000 & 4,70,000 & 2,319        
\end{tabular}
\end{center}
\end{table}
  
  \begin{enumerate}
    \item
	Write down the formula of dependency ratio. \hfill 1
    \item
	What is meant by NRR = 0.983? \hfill 2
    \item  
	Find and compare between the dependency ratios of the cities. \hfill 3
    \item
	Based on data, which city is more comfortable for living? \hfill 4
  \end{enumerate}
  
\end{enumerate}

\end{document}