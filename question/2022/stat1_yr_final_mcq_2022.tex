\documentclass{exam}
%\documentclass[11pt,a4paper]{exam}
\usepackage{amsmath,amsthm,amsfonts,amssymb,dsfont}
\usepackage{ifthen}
\usepackage{enumerate}% http://ctan.org/pkg/enumerate
\usepackage{multicol}



% Accumulate the answers. Unmodified from Phil Hirschorn's answer
% https://tex.stackexchange.com/questions/15350/showing-solutions-of-the-questions-separately/15353
\newbox\allanswers
\setbox\allanswers=\vbox{}

\newenvironment{answer}
{%
    \global\setbox\allanswers=\vbox\bgroup
    \unvbox\allanswers
}%
{%
    \bigbreak
    \egroup
}

\newcommand{\showallanswers}{\par\unvbox\allanswers}
% End Phil's answer


% Is there a better way?
\newcommand*{\getanswer}[5]{%
    \ifthenelse{\equal{#5}{a}}
    {\begin{answer}\thequestion. (a)~#1\end{answer}}
    {\ifthenelse{\equal{#5}{b}}
        {\begin{answer}\thequestion. (b)~#2\end{answer}}
        {\ifthenelse{\equal{#5}{c}}
            {\begin{answer}\thequestion. (c)~#3\end{answer}}
            {\ifthenelse{\equal{#5}{d}}
                {\begin{answer}\thequestion. (d)~#4\end{answer}}
                {\begin{answer}\textbf{\thequestion. (#5)~Invalid answer choice.}\end{answer}}}}}
}

\setlength\parindent{0pt}
%usage \choice{ }{ }{ }{ }
%(A)(B)(C)(D)
\newcommand{\fourch}[5]{
    \par
    \begin{tabular}{*{4}{@{}p{0.23\textwidth}}}
        (a)~#1 & (b)~#2 & (c)~#3 & (d)~#4
    \end{tabular}
    \getanswer{#1}{#2}{#3}{#4}{#5}
}

%(A)(B)
%(C)(D)
\newcommand{\twoch}[5]{
    \par
    \begin{tabular}{*{2}{@{}p{0.46\textwidth}}}
        (a)~#1 & (b)~#2
    \end{tabular}
    \par
    \begin{tabular}{*{2}{@{}p{0.46\textwidth}}}
        (c)~#3 & (d)~#4
    \end{tabular}
    \getanswer{#1}{#2}{#3}{#4}{#5}
}

%(A)
%(B)
%(C)
%(D)
\newcommand{\onech}[5]{
    \par
    (a)~#1 \par (b)~#2 \par (c)~#3 \par (d)~#4
    \getanswer{#1}{#2}{#3}{#4}{#5}
}

\newlength\widthcha
\newlength\widthchb
\newlength\widthchc
\newlength\widthchd
\newlength\widthch
\newlength\tabmaxwidth

\setlength\tabmaxwidth{0.96\textwidth}
\newlength\fourthtabwidth
\setlength\fourthtabwidth{0.25\textwidth}
\newlength\halftabwidth
\setlength\halftabwidth{0.5\textwidth}

\newcommand{\choice}[5]{%
\settowidth\widthcha{AM.#1}\setlength{\widthch}{\widthcha}%
\settowidth\widthchb{BM.#2}%
\ifdim\widthch<\widthchb\relax\setlength{\widthch}{\widthchb}\fi%
    \settowidth\widthchb{CM.#3}%
\ifdim\widthch<\widthchb\relax\setlength{\widthch}{\widthchb}\fi%
    \settowidth\widthchb{DM.#4}%
\ifdim\widthch<\widthchb\relax\setlength{\widthch}{\widthchb}\fi%

% These if statements were bypassing the \onech option.
% \ifdim\widthch<\fourthtabwidth
%     \fourch{#1}{#2}{#3}{#4}{#5}
% \else\ifdim\widthch<\halftabwidth
% \ifdim\widthch>\fourthtabwidth
%     \twoch{#1}{#2}{#3}{#4}{#5}
% \else
%      \onech{#1}{#2}{#3}{#4}{#5}
%  \fi\fi\fi}

% Allows for the \onech option.
\ifdim\widthch>\halftabwidth
    \onech{#1}{#2}{#3}{#4}{#5}
\else\ifdim\widthch<\halftabwidth
\ifdim\widthch>\fourthtabwidth
    \twoch{#1}{#2}{#3}{#4}{#5}
\else
    \fourch{#1}{#2}{#3}{#4}{#5}
\fi\fi\fi}



\begin{document}

\begin{center}
  \bfseries\large
  Sylhet Cadet College

\normalsize
  Year Final Examination - 2022

  Class: XI

  Subject: Statistics First Paper (MCQ)

  Time: 20 minutes \qquad \qquad \qquad \qquad Subject Code: 129   \qquad \qquad \qquad  \qquad Full Marks: 25

%  \normalfont\normalsize
 % 11.45a.m.~--~1.45p.m.
\end{center}

\textbf{Answer all the questions. Each question is worth one (1) mark.}  

\begin{questions}

\question \textbf{In which scale of measurement, zero is regarded as true zero?}
\choice{Nominal scale}{Interval scale}{Ratio scale}{Ordinal scale}{c}

\question \textbf{For which variable, determining number of terms is not possible?}
\choice{Discrete variable}{Continuous variable}{Quantitative variable}{Qualitative variable}{b}

\textbf{Answer the next three question based on the following information.}

\textbf{A farmer collects growth (in cm) o	f 10 plants in a month and finds that \\ $\sum x_i = 7$ and $\sum x_i^2=15$}

\question \textbf{What is the value of $\sum (x_i+4)$?}
\choice{23}{$\sum x_i +4n$}{22}{11}{a}

\question \textbf{What is the value of $\sum (x_i-4)^2$?}
\choice{23}{135}{484}{121}{a}

\question \textbf{If the square of summation is subtracted the sum of square, the value is - }
\choice{-8}{34}{8}{-34}{d}

\question \textbf{Which one is not an example of ratio scale?}
\choice{Room no.}{Income}{Number of accidents}{Weight}{a}

\question \textbf{Which one falls in the category of interval scale?}
\choice {Speed}{Temperature}{Distance}{Film rating}{b}

\question \textbf{The arithmetic mean of first n natural numbers-}
\choice {$\frac{n}{2}$}{$\frac{n+1}{2}$}{$\frac{n^2}{2}$}{$\frac{n^2-1}{2}$}{b}

\question \textbf{When is the relationship $AM = HM = GM$ true?}
\choice {All values are equal}{The values form a geometric progression}{ The values form an arithmetic progression}{All values are distinct}{a}

\question \textbf{In the presence of outlier(s), which measure of central tendency is suitable?}
\choice {Arithmetic mean}{Median}{Quadratic mean}{Power mean}{b}

\question \textbf{If a rate is defined as $R = \frac cd$, where c is constant, then which measure is perfect?}
\choice {Weighted arithmetic mean}{Harmonic mean}{Quadratic mean}{Weighted geometric mean}{b}

\textbf{Answer the next two questions as per the following information.}

42 44 59 64 70 72 74 91 94 are 9 values.

\question \textbf{What is the 50th percentile?}
\choice {64}{70}{72}{71}{b}

\question \textbf{Below which value lie 70 percent values?}
\choice {42}{44}{59}{74}{d}

\question \textbf{Which measure might have more than one value?}
\choice{Arithmetic mean}{Geometric mean}{Quadratic mean}{Mode}{d}

\question \textbf{Above which value lie 30\% observations?}
\choice{3rd Quartile}{Median}{30th Percentile}{70th percentile}{d}

\question \textbf{Arithmetic means of three groups having equal no. of items are 30, 32, and 34. What is the combined mean?}
\choice{30.33}{32.67}{32.00}{33.00}{c}

\question \textbf{How many types of skewness are there?}
\choice {1}{2}{3}{4}{c}

\question \textbf{Which moment may have a negative value?}
\choice{$\mu_4$}{$\mu_3$}{$\mu_2$}{$\mu_2'$}{b}

\question \textbf{Which moment is equivalent to variance?}
\choice {First raw moment around 0}{2nd central moment}{2nd raw moment around median}{First raw moment around arithmetic mean}{b}

\question \textbf{In a right-skewed distribution - }
\choice {Average values are very frequent}{Low values have very low frequency}{High values have very low frequency}{All values have uniform frequency}{c}

\question \textbf{Which one is included in Five Number Summary?}
\choice{2nd Central Moment}{1st Raw Moment}{Median}{Variance}{c}

\question \textbf{What is the value of first central moment?}
\choice {Variance}{Arithmetic mean}{Standard deviation}{0}{d}

\question \textbf{If the first raw moment around 2 is 3, what is the value of $\bar x$?}
\choice {2}{5}{3}{1}{c}

\question \textbf{Which measure does not depend on change of origin?}
\choice {Arithmetic mean}{Standard deviation}{Geometric mean }{Median}{b}

\question \textbf{If the first raw moment around 2 is 3, what is it around 0?}
\choice {3}{0}{6}{5}{d}



%\question \textbf{To complete the song, the last answer should be
%\choice{a}{b}{c}{d}{e} % Invalid answer choice

\end{questions}

%\newpage  %Uncomment to put on new age
\bigskip

%\begin{multicols}{3}
%[
%Answer Key: 
%]
%\showallanswers % Phil Hirschorn
%\end{multicols}


\end{document}
