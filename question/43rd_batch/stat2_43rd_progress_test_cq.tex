\documentclass{article}
\usepackage{geometry}
\usepackage{amsfonts}

\geometry{
legalpaper, total={177.8mm, 290mm},left=20mm,
top=27mm, bottom=27mm,
}

\begin{document}

\begin{table}[h]
\centering
\begin{tabular}{lllll}
\textbf{\large SYLHET CADET COLLEGE} &  &  &  &  \\ \cline{4-5} 
TEST EXAMINATION - 2024 &  & \multicolumn{1}{l|}{} & \multicolumn{1}{l|}{Set} & \multicolumn{1}{l|}{C} \\ \cline{4-5} 
CLASS: XII &  &  &  &  \\ \cline{3-5} 
STATISTICS (CREATIVE)& \multicolumn{1}{l|}{\textbf{Subject Code:}} & \multicolumn{1}{l|}{1} & \multicolumn{1}{l|}{3} & \multicolumn{1}{l|}{0} \\ \cline{3-5} 
SECOND PAPER &  &  &  &  \\
TIME – 2 hours \& 35 minutes &  &  &  &  \\
FULL MARKS – 50 &  &  &  & 
\end{tabular}
\end{table}
%  \normalfont\normalsize
 % 11.45a.m.~--~1.45p.m.

\hrule

\begin{center}
[\textbf{N.B.} – The figures of the right margin indicate full marks. Read the stems carefully and answer the associated questions. Answer any \textbf{FIVE} questions taking at least two from each group.]\\

\end{center}

  \begin{center}
  \textbf{Group--A}
  \end{center}
    \begin{enumerate}
    
     \item
	  \textbf{Events that do not depend on each other are called independent events, and events that cannot occurr simulataneously are called disjoint events.} 
  
  \begin{enumerate}
    \item
	Provide an example of disjoint events, using the set theory. \hfill 1
    \item
	Prove that $P(A\cap \bar B) = P(A) - P(A\cap B)$ \hfill 2
    \item  
	If there are k mutually and exhaustive events, prove $\displaystyle \sum_{i=1}^k P(A_i) = 1$ \hfill 3
    \item
	Prove that two events cannot be simulataneously independent and mutually exclusive. \hfill 4
  \end{enumerate}

   \item
	  \textbf{Ratul and Tomal both have an unbiased die. Both have randomly thrown their dice once. } 
  
  \begin{enumerate}
    \item
	What are equally likely events? \hfill 1
    \item
	If a die is thrown once, what is the probability of getting a prime number? \hfill 2
    \item  
	From the stem, what is the probability that the sum of numbers appearing \\ on the dice is greater than 6. \hfill 3
    \item
	Examine: the probabilities of getting the sum less than 6 and greater than are equal. \hfill 4
  \end{enumerate}
  
   \item
	  \textbf{The probability mass function (pmf) of a football striker scoring no. of hattricks during the course of a league season is given below}
	  
	  \begin{center}
	  $\displaystyle P(x)  = \frac{|2-x|}{k}; x = 0, 1, 2, 3, 4, 5$
	  \end{center}
  
  \begin{enumerate}
    \item
    What is a random variable? \hfill 1
    \item
		Is probability a discrete variable? Explain in brief. \hfill 2
    \item  
	Find the value of k.  \hfill 3
    \item
	Find the probability that the no. of hattricks would be less than the expectation. \hfill 4
  \end{enumerate}
  
     \item
	  \textbf{The probability distributions of demand of mobile phones of two operating systems (OS) Android (X) and iPhone OS (iOS) (Y) are:} 
	  
	    \begin{table}[h]
	    	  \begin{center}
\begin{tabular}{c|c|c|c|c|c}
Demand & 100  & 200  & 300  & 400  & 500  \\ \hline
P(X)   & 0.1  & 0.4  & m    & 0.15 & 0.1  \\ \hline
P(Y)   & 0.09 & 0.45 & 0.32 & 0.11 & 0.03
\end{tabular}
	  \end{center}
\end{table}
  
  \begin{enumerate}
    \item
	What is Expectation? \hfill 1
    \item
	Can Expectation be negative? \hfill 2
    \item  
	Find m from the table. \hfill 3
    \item
	Which OS has higher demand? Analyze. \hfill 4
  \end{enumerate}
  
    \begin{center}
  \textbf{Group--B}
  \end{center}
  

\end{enumerate}

 \vspace{2.5cm}

\begin{center}
\textbf{“We are drowning in information and starving for knowledge.” - \textit{Rutherford David Rogers}}
\end{center}

\end{document}