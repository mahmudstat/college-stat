\documentclass{article}
\usepackage{geometry}
\usepackage{amsfonts}

\geometry{
legalpaper, total={177.8mm, 290mm},left=20mm,
top=27mm, bottom=27mm,
}

\begin{document}

\begin{table}[h]
\centering
\begin{tabular}{lllll}
\textbf{\large SYLHET CADET COLLEGE} &  &  &  &  \\ \cline{4-5} 
PROGRESS TEST EXAMINATION - 2024 &  & \multicolumn{1}{l|}{} & \multicolumn{1}{l|}{Set} & \multicolumn{1}{l|}{A} \\ \cline{4-5} 
CLASS: HSC &  &  &  &  \\ \cline{3-5} 
STATISTICS (CREATIVE)& \multicolumn{1}{l|}{\textbf{Subject Code:}} & \multicolumn{1}{l|}{1} & \multicolumn{1}{l|}{2} & \multicolumn{1}{l|}{9} \\ \cline{3-5} 
 FIRST PAPER &  &  &  &  \\
TIME – 2 hours \& 35 minutes &  &  &  &  \\
FULL MARKS – 50 &  &  &  & 
\end{tabular}
\end{table}
%  \normalfont\normalsize
 % 11.45a.m.~--~1.45p.m.

\hrule

\begin{center}
[\textbf{N.B.} – The figures of the right margin indicate full marks. Read the stems carefully and answer the associated questions. Answer any \textbf{FIVE} questions taking at least two from each group.]\\

\end{center}

  \begin{center}
  \textbf{Group--A}
  \end{center}

  \begin{enumerate}
  
   \item
	  \textbf{Marks obtained by five studnets in statisics out of 15 were 4, 6, 10, 12, and 15. The examiner said, the square of the sum of the marks is greater than the sum of the squares of the marks.} 
  
  \begin{enumerate}
    \item
	What is finite population? \hfill 1
    \item
	Explain quantitative variable with an example. \hfill 2
    \item  
	In the light of the available data, find $\displaystyle \sum_{i=1}^5 (x_i-2x)^2$ \hfill 3
    \item
	Verify the comment of the examiner. \hfill 4
  \end{enumerate}
  
     \item
	  \textbf{A system analyst collected frequencies of a signal at different times. Then he realized due to some unknown noise, 0.5 units got added to all the values. The recorded values are given below:} 
	  
	  \begin{center}
	  10, 12, 15, 14, 12, 16, 20, 16, 18, 11
	  \end{center}
  
  \begin{enumerate}
    \item
	What is change of origin? \hfill 1
    \item
	Does change of origin have an effect on median? \hfill 2
    \item  
	Find $\displaystyle \sum_{i=1}^{10} (X_i-5)$. \hfill 3
    \item
	Determine the summation of the values discarding the noise. \hfill 4
  \end{enumerate}
  
     \item
	  \textbf{A shrimp producer wanted to get an insight into his shrimp production. To do so, he randomly collected weights of different shrimps in his farm.} 
	  

	  \begin{table}[h]
	  \centering
\begin{tabular}{c|c|c|c|c|c}
Weight of shrimp (gm) & 10-20 & 20-30 & 30-40 & 40-50 & 50-60 \\ \hline
Frequency & 5 & 8 & 10 & 9 & 4
\end{tabular}
\end{table}

  \begin{enumerate}
    \item
	What is the primary goal of central tendency?\hfill 1
    \item
	When is Median a better measure of central tendency than Arithmetic Mean and why? \hfill 2
    \item  
	From the stem, find 3rd quartile and explain. \hfill 3
    \item
	Find harmonic mean (HM) and compare with the arithmetic mean (AM) \hfill 4
  \end{enumerate}
  
   \item
	  \textbf{There are only two students in a class IX in a college. In half-yearly exam, the arithmetic and the geometric mean of the marks of those two studnets are 25 and 15, respectively.} 
  
  \begin{enumerate}
    \item
	Write down the formula of Geometric Mean for grouped data. \hfill 1
	\item Prove with an example: $\displaystyle \sum_{i=1}^n (x_i-\bar x) = 0$ \hfill 2
    \item  
	Determine the marks of the students. \hfill 3
    \item
	Is 20 a possible value of the harmonic mean of this data? Explain theoretically and empirically. \hfill 4
  \end{enumerate}

    \begin{center}
  \textbf{Group--B}
  \end{center}
  
 \item
	  \textbf{The first four moments of a dataset were the following:}
	  
	  \begin{center}
	  \textbf{-1, 5, 20, 90}
	  \end{center}
  
  \begin{enumerate}
    \item
	What is raw moment? \hfill 1
    \item
	What is the standard deviation of the data in the stem? \hfill 2
    \item  
	Determine the third central moment. \hfill 3
    \item
	Comment on the kurtosis of the given data. \hfill 4
  \end{enumerate}
  
   \item
	  \textbf{The heights of the trees of a certain species that were planted at around the same time in a park were examined by an analyst, hired by the park authority to check for any abnormality. He randomly observed 10 trees; the values (in cm) obtained are given below:} 
	  
	  	  \begin{center}
	  \textbf{200, 190, 185, 210, 220, 200, 205, 207, 230, 195}
	  \end{center}
  
  \begin{enumerate}
    \item
	What is five number summary? \hfill 1
    \item
	Which measures are shown on a Box \& Whisker plot? \hfill 2
    \item  
	Display the data on a box plot. \hfill 3
    \item
	Taking a look at the drawn box plot, comment on the symmetry of the data. \hfill 4
  \end{enumerate}
  
   \item
	  \textbf{Statement} 
  
  \begin{enumerate}
    \item
	Question \hfill 1
    \item
	Question \hfill 2
    \item  
	Question \hfill 3
    \item
	Question \hfill 4
  \end{enumerate}

\end{enumerate}
\end{document}

