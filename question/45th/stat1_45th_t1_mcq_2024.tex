\documentclass{exam}
%\documentclass[11pt,a4paper]{exam}
\usepackage{amsmath,amsthm,amsfonts,amssymb,dsfont}
\usepackage{ifthen}
\usepackage{geometry}
\geometry{
legalpaper, total={177.8mm, 290mm},left=20mm,
top=7mm, bottom=27mm,
}
\usepackage{enumerate}% http://ctan.org/pkg/enumerate
\usepackage{multicol}
\usepackage{hhline}
\usepackage[table]{xcolor}


% Accumulate the answers. Unmodified from Phil Hirschorn's answer
% https://tex.stackexchange.com/questions/15350/showing-solutions-of-the-questions-separately/15353
\newbox\allanswers
\setbox\allanswers=\vbox{}

\newenvironment{answer}
{%
    \global\setbox\allanswers=\vbox\bgroup
    \unvbox\allanswers
}%
{%
    \bigbreak
    \egroup
}

\newcommand{\showallanswers}{\par\unvbox\allanswers}
% End Phil's answer


% Is there a better way?
\newcommand*{\getanswer}[5]{%
    \ifthenelse{\equal{#5}{a}}
    {\begin{answer}\thequestion. (a)~#1\end{answer}}
    {\ifthenelse{\equal{#5}{b}}
        {\begin{answer}\thequestion. (b)~#2\end{answer}}
        {\ifthenelse{\equal{#5}{c}}
            {\begin{answer}\thequestion. (c)~#3\end{answer}}
            {\ifthenelse{\equal{#5}{d}}
                {\begin{answer}\thequestion. (d)~#4\end{answer}}
                {\begin{answer}\textbf{\thequestion. (#5)~Invalid answer choice.}\end{answer}}}}}
}

\setlength\parindent{0pt}
%usage \choice{ }{ }{ }{ }
%(A)(B)(C)(D)
\newcommand{\fourch}[5]{
    \par
    \begin{tabular}{*{4}{@{}p{0.23\textwidth}}}
        (a)~#1 & (b)~#2 & (c)~#3 & (d)~#4
    \end{tabular}
    \getanswer{#1}{#2}{#3}{#4}{#5}
}

%(A)(B)
%(C)(D)
\newcommand{\twoch}[5]{
    \par
    \begin{tabular}{*{2}{@{}p{0.46\textwidth}}}
        (a)~#1 & (b)~#2
    \end{tabular}
    \par
    \begin{tabular}{*{2}{@{}p{0.46\textwidth}}}
        (c)~#3 & (d)~#4
    \end{tabular}
    \getanswer{#1}{#2}{#3}{#4}{#5}
}

%(A)
%(B)
%(C)
%(D)
\newcommand{\onech}[5]{
    \par
    (a)~#1 \par (b)~#2 \par (c)~#3 \par (d)~#4
    \getanswer{#1}{#2}{#3}{#4}{#5}
}

\newlength\widthcha
\newlength\widthchb
\newlength\widthchc
\newlength\widthchd
\newlength\widthch
\newlength\tabmaxwidth

\setlength\tabmaxwidth{0.96\textwidth}
\newlength\fourthtabwidth
\setlength\fourthtabwidth{0.25\textwidth}
\newlength\halftabwidth
\setlength\halftabwidth{0.5\textwidth}

\newcommand{\choice}[5]{%
\settowidth\widthcha{AM.#1}\setlength{\widthch}{\widthcha}%
\settowidth\widthchb{BM.#2}%
\ifdim\widthch<\widthchb\relax\setlength{\widthch}{\widthchb}\fi%
    \settowidth\widthchb{CM.#3}%
\ifdim\widthch<\widthchb\relax\setlength{\widthch}{\widthchb}\fi%
    \settowidth\widthchb{DM.#4}%
\ifdim\widthch<\widthchb\relax\setlength{\widthch}{\widthchb}\fi%

% These if statements were bypassing the \onech option.
% \ifdim\widthch<\fourthtabwidth
%     \fourch{#1}{#2}{#3}{#4}{#5}
% \else\ifdim\widthch<\halftabwidth
% \ifdim\widthch>\fourthtabwidth
%     \twoch{#1}{#2}{#3}{#4}{#5}
% \else
%      \onech{#1}{#2}{#3}{#4}{#5}
%  \fi\fi\fi}

% Allows for the \onech option.
\ifdim\widthch>\halftabwidth
    \onech{#1}{#2}{#3}{#4}{#5}
\else\ifdim\widthch<\halftabwidth
\ifdim\widthch>\fourthtabwidth
    \twoch{#1}{#2}{#3}{#4}{#5}
\else
    \fourch{#1}{#2}{#3}{#4}{#5}
\fi\fi\fi}


\begin{document}

\begin{table}[h]
\centering
\begin{tabular}{lllll}
\textbf{\large SYLHET CADET COLLEGE} &  &  &  &  \\ \cline{4-5} 
FIRST TERM-END EXAMINATION - 2024 &  & \multicolumn{1}{l|}{} & \multicolumn{1}{l|}{Set} & \multicolumn{1}{l|}{A} \\ \cline{4-5} 
CLASS: XI &  &  &  &  \\ \cline{3-5} 
MCQ and Short QUESTIONS & \multicolumn{1}{l|}{\textbf{Subject Code:}} & \multicolumn{1}{l|}{1} & \multicolumn{1}{l|}{2} & \multicolumn{1}{l|}{9} \\ \cline{3-5} 
STATISTICS FIRST PAPER &  &  &  &  \\
TIME – 25 minutes &  &  &  &  \\
FULL MARKS – 25 &  &  &  & 
\end{tabular}
\end{table}
%  \normalfont\normalsize
 % 11.45a.m.~--~1.45p.m.
\hrule

\begin{center}
[N.B. – Answer all the questions. Each question carries ONE mark.]\\

  
  \textbf{Candidates are asked not to leave any mark or spot on the question paper.}
\end{center}
\begin{questions}

\question \textbf{If $\displaystyle \sum_{i=1}^{20} x_i^2=20$ and
$\displaystyle \sum_{i=1}^{20} x_i=30$, what is the value of 
$\displaystyle \sum_{i=1}^{20} x_i^2 + \sum_{i=1}^{20} x_i + 100$?}
\choice{130}{200}{150}{2130}{c}

\question \textbf{Which of the following is a continuous variable?}
\choice{Number of goals}{Natural number}{Summation of Fibonacci series}
{Success rate}{d}

\question \textbf{If $x_1=4$, $x_2=1$, $x_3=-2$, and $x_4=3$, find 
$\displaystyle \sum_{i=1}^4 (x_i^2 + 3)$?}  
\choice{40}{50}{42}{56}{c} 

\textbf{Answer the next three questions based on the following information.}

The values of $x_i$ and $f_i$ are given below:

\begin{table}[h]
\centering
\begin{tabular}{c|c|c|c|c}
$x_i$ & 2     & 4     & 6     & 8     \\ \hline
$f_i$ & 2     & 2     & 5     & 4    
\end{tabular}
\end{table}

\question \textbf{Find $\displaystyle \sum_{i=1}^4 f_i x_i$.}  
\choice{50}{74}{56}{60}{b}  

\question \textbf{Compute $\displaystyle \sum_{i=1}^4 f_i x_i^2$.}  
\choice{256}{274}{476}{300}{c}  

\question \textbf{Determine $\displaystyle \sum_{i=1}^4 f_i (x_i-5)^2$.}  
\choice{61}{48}{52}{58}{a}  

\question \textbf{Which is an advantage of primary data?}

i. Specific to the study \\
ii. More reliable \\
iii. Less time-consuming

\textbf{Which one is correct?}

\choice{i and ii}{i and iii}{ii and iii}{i, ii and iii}{a}




%--------Group Starts
\textbf{Answer the next THREE questions based on the following information.}

The weights of 120 fruits were recorded and this frequency distribution was constructed.

\begin{table}[h]
\centering
\begin{tabular}{c|c|c|c|c}
\begin{tabular}[c]{@{}c@{}}Weight (grams)\end{tabular} & 0-50 & 50-100 & 100-150 & 150-200 \\ \hline
No. of Fruits & 30 & 35 & 25 & 30
\end{tabular}
\end{table}

\question \textbf{How many fruits weigh at least 100 grams?}
\choice{55}{50}{60}{65}{a}

\question \textbf{How many fruits weigh less than 100 grams?}
\choice{68}{70}{65}{50}{c}

\question \textbf{What percent of fruits weigh between 50 and 150 grams?}
\choice{50\%}{55\%}{60\%}{75\%}{c}
%--------Group Ends

\question \textbf{Which of the following represents primary data?}

i. A scientist collects soil samples for analysis \\
ii. Data compiled in a textbook \\
iii. A business owner surveys customers directly

\textbf{Which one is correct?}

\choice{i and iii}{i and ii}{ii and iii}{i, ii, and iii}{a}

\textbf{Answer the next two questions based on the following information}

\begin{center}
\begin{table}[h]
\centering
\begin{tabular}{c|c|c|c|c}
Class Interval & <10     & 10-20     & 20-30     & 30-40     \\ \hline
Frequency & 6     & 3     & 7     & 4    
\end{tabular}
\end{table}
\end{center}

\question \textbf{What is relative frequency of the class with the highest
  frequency?}
\choice{0.25}{0.45}{0.40}{0.35}{d}

\question \textbf{Which curve is suitable for }
\choice{Histogram}{Bar Diagram}{Pie Chart}{Ogive}{d}

\question \textbf{Which measure of central tendency is suitable for 
qualitative variable?}
\choice{Arithmetic Mean}{Harmonic Mean}{Quadratic Mean}{Mode}{d}

\question \textbf{If $\sum (x_i-k)=0$, what is the value of k?}
\choice{$n$}{$\bar x$}{$x$}{$n \bar x$}{b}

\question \textbf{Give an example of a continuous variable.} 
\underline{\hspace{5cm}}

\question \textbf{How many measurement scales are there?} 
\underline{\hspace{5cm}}

\item \textbf{After expansion, what does $\displaystyle \sum_{i=1}^n 
\left( ax_i-b \right)$ become?}
\underline{\hspace{5cm}}

\item \textbf{What is change of scale?} 
\underline{\hspace{5cm}}

\item \textbf{What is the primary use of a histogram?} 
\underline{\hspace{5cm}}

\item \textbf{What information can you gather from a pie chart?} 
\underline{\hspace{5cm}}

\item \textbf{What is the purpose of a frequency distribution?}
\underline{\hspace{5cm}}

\item \textbf{What is the primary goal of central tendency?}
\underline{\hspace{5cm}}

\item \textbf{Find median: $7,  2,  4,  5,  6, 10$}
\underline{\hspace{5cm}}

\item \textbf{What is a variable?} 
\underline{\hspace{5cm}}

\end{questions}

 \vspace{2.5cm}

\begin{center}
"It is a capital mistake to theorize before one has data." -- Sir Arthur Conan Doyle
\end{center}

\pagebreak
%\newpage  %Uncomment to put on new age
\bigskip

\begin{multicols}{3}
[
Answer Key
]
\showallanswers % Phil Hirschorn
\end{multicols}


\end{document}