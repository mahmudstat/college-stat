\documentclass{article}
\usepackage{geometry}
\usepackage{amsfonts}

\geometry{
legalpaper, total={177.8mm, 290mm},left=20mm,
top=7mm, bottom=27mm,
}

\begin{document}

\begin{table}[h]
\centering
\begin{tabular}{lllll}
\textbf{\large SYLHET CADET COLLEGE} &  &  &  &  \\ \cline{4-5} 
FIRST TERM-END EXAMINATION - 2024 &  & \multicolumn{1}{l|}{} & 
\multicolumn{1}{l|}{Set} & \multicolumn{1}{l|}{A} \\ \cline{4-5} 
CLASS: XI &  &  &  &  \\ \cline{3-5} 
STATISTICS (CREATIVE)& \multicolumn{1}{l|}{\textbf{Subject Code:}} & 
\multicolumn{1}{l|}{1} & \multicolumn{1}{l|}{2} & \multicolumn{1}{l|}{9} \\ \cline{3-5} 
 FIRST PAPER &  &  &  &  \\
TIME – 2 hours \& 35 minutes &  &  &  &  \\
FULL MARKS – 50 &  &  &  & 
\end{tabular}
\end{table}
%  \normalfont\normalsize
 % 11.45a.m.~--~1.45p.m.

\hrule

\begin{center}
[\textbf{N.B.} – The figures of the right margin indicate full marks. Read 
the stems carefully and answer the associated questions. Answer all the
\textbf{FIVE} questions.]\\
\end{center}

%\begin{center}
% \textbf{Group  - A}

%\end{center}
  \begin{enumerate}

\item
\textbf{The monthly sales and expenses (in thousand BDT) of five retail 
stores are given below:}

\begin{table}[h]
\centering
\begin{tabular}{c|ccccc}
Store & A & B & C & D & E \\ \hline
Sales (X) & 50 & 65 & 40 & 70 & 55 \\ \hline
Expenses (Y) & 30 & 45 & 25 & 50 & 35
\end{tabular}
\end{table}

\begin{enumerate}
  \item What is a sample? \hfill 1
  \item Differentiate between discrete and continuous variable. \hfill 2
    \item 
    Calculate $\displaystyle \sum_{i=1}^5 (x_i + y_i)$ \hfill 3
    \item 
    Verify whether $\displaystyle \sum_{i=1}^5 (3x_i - 2y_i) = 3 
    \sum_{i=1}^5 x_i - 2 \sum_{i=1}^5 y_i$ holds true. \hfill 4
\end{enumerate}

     \item
	  \textbf{The following table tracks the number of individuals who sleep 
	  within specific hourly intervals. }
	  
\begin{table}[h]
  \centering
\begin{tabular}{c|c|c|c|c|c|c|c}
Hours of Sleep (per night) & 4-5   & 5-6   & 6-7   & 7-8   & 8-9   & 9-10  & 10+   \\ \hline
Number of Individuals      & 12    & 20    & 25    & 30    & 18    & 8     & 7     
\end{tabular}
\end{table}


  \begin{enumerate}
    \item  What is a histogram used for? \hfill 1
    \item Relate histogram and stem and leaf plot \hfill 2
      \item 
	Draw an Ogive from the data provided and explain. \hfill 3
    \item
	Write five useful insights about the data combining information from 
	the Ogive and the table. \hfill 4
  \end{enumerate}

\item
\textbf{The number of hours worked (in a week) and weekly earnings 
(in dollars) of some employees \\ are recorded as follows:}

\begin{table}[!h]
\begin{center}
\begin{tabular}{l|l|l|l|l}
Hours Worked (x)  & 40 & 35 & 50 & 30 \\ \hline
Weekly Earnings (y) & 400  & 350  & 500 & 300 \\ 
\end{tabular}
\end{center}
\end{table}

\begin{enumerate}
    \item What is univariate data? \hfill 1
    \item Explain change of origin and scale with an example. \hfill 2
    \item
    Are, in the stem, $\displaystyle \sum_{i=1}^{n} 
    \sum_{i=1}^{n} x_iy_j = \sum_{i=1}^{n} x_iy_i$? Vindicate \hfill 3
    \item
    Using the data, prove that the sum of squares is unequal 
    to the square of the sum of numbers. \hfill 4
\end{enumerate}

  \item
  \textbf{The ages of 20 participants in a fitness program were recorded and 
  found to be as follows:}
  \begin{center}
  25, 30, 28, 35, 40, 38, 26, 32, 36, 31 \\
  27, 33, 29, 41, 42, 37, 34, 39, 43, 45 \\
  \end{center}

  
  \begin{enumerate}
     \item If two class intervals of a frequency distribution are (10-30) and 
 (30-50), what is the width of \\ class interval? \hfill 1
  \item For a pie chart, how are the angles calculated? \hfill 2
    \item  
	 Create a frequency distribution and interpret. \hfill 3
    \item
	Create a Histogram from the data and explain. If the no. of 
	classes were fewer, how would the \\ pattern of the distribution shift? \hfill 4
  \end{enumerate}
  
  \vspace{1cm}
  
  \hfill Continue on the next page ...
  
  \newpage
     \item
	  \textbf{Mean monthly salaries of employees of two companies A \& B are tk. 
	  65,000 and tk. 75,000. \\ The combined arithmetic mean (AM) is tk. 
	  71,000 and number of employees in \\ the company A is 20.} 
	  
  \begin{enumerate}
    \item
	Write down the formula of combined AM for k groups. \hfill 1
    \item
	What is the combined AM of two data sets with AM 35 and 45 and number of 
	values equal? \hfill 2
    \item  
	How many employees are there in the company B? \hfill 3
    \item
	Salary of an employee of company A was recorded as tk. 60,000 in place of
	65,000. \\ What is the new AM of company A. Also find the corrected 
	combined AM.\hfill 4
  \end{enumerate}
  
  \vspace{3cm}
  
  \begin{center}
  
  “\textit{Without data, you’re just another person with an opinion.}” ---
  William Edwards Deming
  
  \end{center}

  
\end{enumerate}
\end{document}