\documentclass{exam}
%\documentclass[11pt,a4paper]{exam}
\usepackage{amsmath,amsthm,amsfonts,amssymb,dsfont}
\usepackage{ifthen}
\usepackage{setspace}
\usepackage{geometry}
\geometry{
legalpaper, total={177.8mm, 290mm},left=20mm,
top=10mm, bottom=20mm,
}
\usepackage{enumerate}% http://ctan.org/pkg/enumerate
\usepackage{multicol}
\usepackage{hhline}
\usepackage[table]{xcolor}


% Accumulate the answers. Unmodified from Phil Hirschorn's answer
% https://tex.stackexchange.com/questions/15350/showing-solutions-of-the-questions-separately/15353
\newbox\allanswers
\setbox\allanswers=\vbox{}

\newenvironment{answer}
{%
    \global\setbox\allanswers=\vbox\bgroup
    \unvbox\allanswers
}%
{%
    \bigbreak
    \egroup
}

\newcommand{\showallanswers}{\par\unvbox\allanswers}
% End Phil's answer


% Is there a better way?
\newcommand*{\getanswer}[5]{%
    \ifthenelse{\equal{#5}{a}}
    {\begin{answer}\thequestion. (a)~#1\end{answer}}
    {\ifthenelse{\equal{#5}{b}}
        {\begin{answer}\thequestion. (b)~#2\end{answer}}
        {\ifthenelse{\equal{#5}{c}}
            {\begin{answer}\thequestion. (c)~#3\end{answer}}
            {\ifthenelse{\equal{#5}{d}}
                {\begin{answer}\thequestion. (d)~#4\end{answer}}
                {\begin{answer}\textbf{\thequestion. (#5)~Invalid answer choice.}\end{answer}}}}}
}

\setlength\parindent{0pt}
%usage \choice{ }{ }{ }{ }
%(A)(B)(C)(D)
\newcommand{\fourch}[5]{
    \par
    \begin{tabular}{*{4}{@{}p{0.23\textwidth}}}
        (a)~#1 & (b)~#2 & (c)~#3 & (d)~#4
    \end{tabular}
    \getanswer{#1}{#2}{#3}{#4}{#5}
}

%(A)(B)
%(C)(D)
\newcommand{\twoch}[5]{
    \par
    \begin{tabular}{*{2}{@{}p{0.46\textwidth}}}
        (a)~#1 & (b)~#2
    \end{tabular}
    \par
    \begin{tabular}{*{2}{@{}p{0.46\textwidth}}}
        (c)~#3 & (d)~#4
    \end{tabular}
    \getanswer{#1}{#2}{#3}{#4}{#5}
}

%(A)
%(B)
%(C)
%(D)
\newcommand{\onech}[5]{
    \par
    (a)~#1 \par (b)~#2 \par (c)~#3 \par (d)~#4
    \getanswer{#1}{#2}{#3}{#4}{#5}
}

\newlength\widthcha
\newlength\widthchb
\newlength\widthchc
\newlength\widthchd
\newlength\widthch
\newlength\tabmaxwidth

\setlength\tabmaxwidth{0.96\textwidth}
\newlength\fourthtabwidth
\setlength\fourthtabwidth{0.25\textwidth}
\newlength\halftabwidth
\setlength\halftabwidth{0.5\textwidth}

\newcommand{\choice}[5]{%
\settowidth\widthcha{AM.#1}\setlength{\widthch}{\widthcha}%
\settowidth\widthchb{BM.#2}%
\ifdim\widthch<\widthchb\relax\setlength{\widthch}{\widthchb}\fi%
    \settowidth\widthchb{CM.#3}%
\ifdim\widthch<\widthchb\relax\setlength{\widthch}{\widthchb}\fi%
    \settowidth\widthchb{DM.#4}%
\ifdim\widthch<\widthchb\relax\setlength{\widthch}{\widthchb}\fi%

% These if statements were bypassing the \onech option.
% \ifdim\widthch<\fourthtabwidth
%     \fourch{#1}{#2}{#3}{#4}{#5}
% \else\ifdim\widthch<\halftabwidth
% \ifdim\widthch>\fourthtabwidth
%     \twoch{#1}{#2}{#3}{#4}{#5}
% \else
%      \onech{#1}{#2}{#3}{#4}{#5}
%  \fi\fi\fi}

% Allows for the \onech option.
\ifdim\widthch>\halftabwidth
    \onech{#1}{#2}{#3}{#4}{#5}
\else\ifdim\widthch<\halftabwidth
\ifdim\widthch>\fourthtabwidth
    \twoch{#1}{#2}{#3}{#4}{#5}
\else
    \fourch{#1}{#2}{#3}{#4}{#5}
\fi\fi\fi}


\begin{document}

\begin{table}[h]
\centering
\begin{tabular}{lllll}
\textbf{\large SYLHET CADET COLLEGE} &  &  &  &  \\ \cline{4-5} 
SPECIAL EXAMINATION - 2024 &  & \multicolumn{1}{l|}{} & \multicolumn{1}{l|}{Set} & \multicolumn{1}{l|}{A} \\ \cline{4-5} 
CLASS: XI &  &  &  &  \\ \cline{3-5} 
SAQ and Creative QUESTIONS & \multicolumn{1}{l|}{\textbf{Subject Code:}} & \multicolumn{1}{l|}{1} & \multicolumn{1}{l|}{2} & \multicolumn{1}{l|}{9} \\ \cline{3-5} 
STATISTICS &  &  &  &  \\
TIME – 01 hour &  &  &  &  \\
FULL MARKS – 30 &  &  &  & 
\end{tabular}
\end{table}
%  \normalfont\normalsize
 % 11.45a.m.~--~1.45p.m.
\hrule

\begin{center}

  \textbf{Candidates are asked not to leave any mark or spot on the question paper.}
\end{center}

\textbf{Short Questions}

\onehalfspacing % Set line spacing to 1.5

\begin{enumerate}

\item Expand $\displaystyle \sum_{i=1}^n \left( ax_i-b \right)$. \hfill 3

\item What is change of origin? Show an example.  \hfill 2

    \item
	$x_1=2, x_2=-3, x_3=7, x_4=12.$ 
	
	Find the values of the following: \hfill $2 \times 1.5 = 3$
	
	i) $\displaystyle \sum_{i=1}^3 x_i$ 
	ii) $\displaystyle \sum_{i=1}^4 x_i^2$

	\item \textbf{Write down the scales of measurement of the following variables.} \hfill $4 \times 0.5 = 2$

	Gender, Religion, Temperature, Income group (Lower class, Low, Middle, High)



 \vspace{0.5cm}
 
 \textbf{Creative Questions}

 \item
	  \textbf{A set of values and their respective frequencies are given below:}
	  
\begin{table}[h]
\centering
\begin{tabular}{c|c|c|c}
$x_i$ & 2 & 4 & 8 \\ \hline
$f_i$ & 5 & 4 & 6
\end{tabular}
\end{table}
  
  \begin{enumerate}
    \item
	Is $X_i$ discrete? \hfill 1
    \item
	Find $\displaystyle \sum_{i=1}^2 f_ix_i$ \hfill 2
    \item  
	Estimate $\displaystyle \sum_{i=1}^3 f_ix_i^2$ \hfill 3
    \item
	Are $\displaystyle \sum_{i=1}^3 f_i^2x_i^2$ and $\displaystyle (\sum_{i=1}^3 f_ix_i)^2$ equal? Verify. \hfill 4
  \end{enumerate}

 \item
	  \textbf{$\displaystyle  \sum_{i=1}^m \sum_{j=1}^n (x_i+y_j)$} is a notation
	  making use of double summation. 
  
  \begin{enumerate}
    \item
	What is $\displaystyle \sum_{i=1}^n (a)$, where a is a constant? \hfill 1
    \item
	Why are two suffixes (i and j) used here? \hfill 2
    \item  
	Reduce the expression to single summations. \hfill 3
    \item
	Evaluate the expression if X = 7, 10, 23 and Y = 10, 15, 16 \hfill 4
  \end{enumerate}


\end{enumerate}

 \vspace{2.5cm}

\begin{center}
An approximate answer to the right problem is worth a good deal more than an exact answer to an approximate problem. – John Tukey.
\end{center}

%\pagebreak
%\newpage  %Uncomment to put on new age
%\bigskip

%\begin{multicols}{3}
%[
%Answer Key
%]
%\showallanswers % Phil Hirschorn
%\end{multicols}


\end{document}